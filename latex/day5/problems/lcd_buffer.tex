\section{LCD}
Nun soll das Display wie ein Bildschirm angesteuert werden können.
Dazu wirst du einen Framebuffer verwenden, der den Bildschirminhalt repräsentiert.
In jeder Iteration wird der Framebuffer zunächst vollständig vorbereitet und dann am Stück übertragen.

Die Vorlage enthält bereits ein Array \lstinline{lcd_buffer}, das als Framebuffer fungieren soll.

Implementiere folgende Funktionen:
\lstinputlisting{problems/listings/lcd_buffer.c}

\hints{
	\item Ein Testprogramm steht bereits zur Verfügung, welches ein Schachbrettmuster auf dem Display ausgibt -- wenn deine Funktionen komplett und korrekt implementiert sind.
	\item Achte auch hier darauf, dass das Display zunächst per Befehl eingeschaltet werden muss (\emph{Display on}).
	\item Für die Funktion \lstinline{lcd_drawPixel} werden die bitweisen Operationen AND und OR sowie der Shift-Operator benötigt.
	Eine Skizze kann hier sinnvoll sein, um sich vorzustellen, wie die bitweisen Operationen arbeiten.
	\item Wenn du später bewegte Dinge visualisierst, kann es sein, dass das Display flimmert.
	Hier hilft es, entweder die Anzeige nur zu verändern, wenn sich tatsächlich etwas bewegt oder das Display vor dem Schreiben des Framebuffers aus- (\emph{Display off}) und danach wieder einzuschalten (\emph{Display on}).
}
