\section{Aufzugsimulator}
In dieser Aufgabe soll ein Grundgerüst für den in der Vorlesung vorgestellten Aufzugsimulator geschaffen werden.
Diese Aufgabe wird in den folgenden Übungen erweitert.
Falls du feststellst, dass dir im Moment die Zeit dafür fehlt, kannst du diese Aufgaben auch nach hinten verschieben und als ausführliche Klausurvorbereitung nutzen.

\subsection{Klasse Person}
Implementiere die Klasse \texttt{Person}, die eine Person mit einem gewünschten Zielstockwerk darstellt.
Füge allen Konstruktoren und Destrukoren eine Ausgabe auf die Konsole hinzu, um später den Lebenszyklus der Objekte besser nachvollziehen zu können.

\begin{lstlisting}
class Person {
public:
	Person(int destinationFloor);		// create a person with given destination
	Person(const Person& other);		// copy constructor
	~Person();								// destructor
	int getDestinationFloor() const;	// get the destination floor of this person
private:
	int destinationFloor;				// destination floor of this person
};
\end{lstlisting}

\subsection{Klasse Elevator}
Implementiere die Klasse \texttt{Elevator}, die einen Aufzug mit einer beliebigen Anzahl an Personen darstellt.

Wenn sich der Aufzug bewegt, solltest du die verbrauchte Energie bei einer Bewegung sinnvoll anpassen.
Addiere beispielsweise den Betrag der Differenz zwischen dem aktuellen und dem Zielstockwerk hinzu.

\begin{lstlisting}
class Elevator {
public:
	Elevator();	// create an elevator at floor 0, no people inside and 0 energy consumed
	int getFloor();					// get number of floor the elevator is currently at
	double getEnergyConsumed();	// get consumed energy
	void moveToFloor(int floor);	// move the elevator to given floor (consumes energy)
	int getNumPeople();				// get number of people in Elevator
	Person getPerson(int i);		// get i-th person in Elevator
	void addPeople(std::vector<Person> people);	// add people to Elevator
	std::vector<Person> removeArrivedPeople();	// remove people which arrived
private:
	int currentFloor;								// current floor number
	std::vector<Person> containedPeople;	// people currently in elevator
	double energyConsumed;						// energy consumed
};
\end{lstlisting}

Um die Klasse \texttt{std::vector} aus der Standardbibliothek zu nutzen musst du noch den Systemheader \texttt{vector} einbinden.
Der Container \texttt{std::vector} kapselt ein Array und stellt eine ähnliche Funktionalität wie Javas \texttt{Vector} Klasse bereit.
Der Typ in spitzen Klammern (\texttt{<Person>} in \texttt{std::vector<Person>}) ist ein Template-Parameter und besagt, dass in dem Container \texttt{Person}-Objekte gespeichert werden sollen.

Folgende Funktion der Klasse \texttt{std::vector} könnten dir von Nutzen sein. Weitere findest du z.B. unter \url{http://www.cplusplus.com/reference/vector/vector/}. Der Typ \texttt{size\_type} ist der größtmöglichhe vorzeichenloser Integer, die die verwendete Plattform unterstützt.

\begin{lstlisting}
size_type size() const;						// get size of the vector
reference at(size_type n);					// get the i-th element of the vector
void push_back(const value_type& val);	// add an element to the vector
void clear();									// remove all elements from the vector
\end{lstlisting}

\hints{
	\item Da \texttt{containedPeople} leer initialisiert werden soll, brauchst du dafür keinen expliziten Aufruf in der Initialisierung.
	\item Um die Leute aussteigen zu lassen, die an ihrem Zielstockwerk angekommen sind, erstelle in der Methode zwei temporäre \texttt{std::vector}-Container \texttt{stay} und \texttt{arrived}.
	Iteriere nun über alle Leute im Aufzug und prüfe, ob das Zielstockwerk der Person mit dem aktuellen Stockwerk des Aufzugs übereinstimmt.
	Wenn ja, lasse die Person aussteigen, indem du sie zu der \texttt{arrived}-Liste mittels \texttt{push\_back()} hinzufügst.
	Andernfalls muss die Person im Aufzug verbleiben (\texttt{stay}-Liste).
	Gib am Ende die arrived-Liste zurück, und ersetze \texttt{containedPeople} durch \texttt{stay}.
}


\subsection{Klasse Floor}
Implementiere die Klasse \texttt{Floor}, die ein Stockwerk mit einer beliebigen Anzahl an wartenden Personen darstellt.

\begin{lstlisting}
class Floor {
public:
	int getNumPeople();							// get the number of people on this floor
	Person getPerson(int i);					// get the i-th person on this floor
	void addWaitingPerson(Person h);			// add a person to this floor
	std::vector<Person> removeAllPeople();	// remove all persons from this floor
private:
	std::vector<Person> containedPeople;	// persons on this floor
};
\end{lstlisting}

\subsection{Klasse Building}
Schreibe eine Klasse \texttt{Building}, die einen Aufzug besitzt der sich zwischen einer definierbaren Menge an Stockwerken bewegt und Personen befördert.

\begin{lstlisting}
class Building {
public:
	Building(int numberOfFloors);	// create a Building with given number of floors
	int getNumOfFloors();			// get number of floors
	Floor& getFloor(int floor);	// get a certain floor
	Elevator& getElevator();		// get the elevator
private:
	std::vector<Floor> floors;		// floors of this building
	Elevator elevator;				// the elevator
};
\end{lstlisting}


\subsection{Komfortfunktionen}
Erweitre die Klasse \texttt{Building} um folgende \texttt{public} Funktionen, um die Benutzung des Simulators von außen zu vereinfachen und lange Aufrufketten wie
\begin{lstlisting}
	b.getElevator().addPeople(b.getFloor(b.getElevator().getFloor()).removeAllPeople());
\end{lstlisting}

zu vermeiden (\emph{Law of Demeter}\footnote{\url{https://en.wikipedia.org/wiki/Law_of_Demeter}}).
Der Simulator sollte nur mit Methoden der Klasse \texttt{Building} kommunizieren.

\begin{lstlisting}
void letPeopleIn();									// let people on current floor into elevator
void moveElevatorToFloor(int i);					// move the elevator to a given floor
void addWaitingPerson(int floor, Person p);	// add a person to a given floor
std::vector<Person> removeArrivedPeople();	// remove people which arrived at their destination from the elevator on the current floor
\end{lstlisting}

\subsection{Beförderungsstrategie}
Teste deine Implementation.
Erstelle dazu zunächst ein Gebäude und füge einige Personen hinzu.

\begin{lstlisting}
	Building b(3);
	b.addWaitingPerson(0, Person(2)); // person in floor 0 wants to floor 2
	b.addWaitingPerson(1, Person(0)); // person in floor 1 wants to floor 0
	b.addWaitingPerson(2, Person(0)); // person in floor 2 wants to floor 0
\end{lstlisting}

Implementiere nun folgende Beförderungsstrategie.
Diese sehr einfache (und ineffiziente) Strategie fährt alle Stockwerke nacheinander ab, sammelt die Leute ein und befördert sie jeweils zu ihren Zielstockwerken.

\begin{algorithm}[H]
 \SetAlgoLined
 \For{Floor floor \textbf{in} Building}{
   Move \texttt{elevator} to Floor \texttt{floor};\\
   Let all people on \texttt{floor} into \texttt{elevator};\\
   
  \While{elevator has people} {
    Move Elevator to destination Floor of first Person in Elevator; \\
    Remove arrived people; \\
  }
 }
\end{algorithm}

Gib am Ende auch die verbrauchte Energie aus.
Schau dir die Ausgabe genau an und versuche nachzuvollziehen, warum Personen so oft kopiert werden.
Denke daran, dass diese bei einer Übergabe als Argument kopiert werden.
