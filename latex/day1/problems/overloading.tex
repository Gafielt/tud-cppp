\section{Operatorüberladung}
In C++ besteht die Möglichkeit, Operatoren wie \textbf{+} (\lstinline{operator+}), \textbf{*} (\lstinline{operator*}),\dots zu überladen.
Man kann selber spezifizieren, was beim Verknüpfen von Objekten mit einem Operator geschehen soll, um zum Beispiel den Quellcode übersichtlicher zu gestalten.
Du hast bereits das Objekt \lstinline{std::cout} der Klasse \lstinline{std::ostream} kennengelernt, welche den $<<$-Operator überlädt, um Ausgaben von \lstinline{std::string}, \lstinline{int},\dots komfortabel zu tätigen.
In dieser Aufgabe sollst du eine eigene Vektor-Klasse schreiben und einige Operatoren überladen.

\hints{
	\item Am Tag 4 soll dieser Vektor um weitere Funktionen erweitern werden.
	Falls du mit dieser Aufgabe bis dahin nicht fertig sein solltest, kannst du natürlich auf die Musterlösung zurückgreifen.
    
    \item Ausführliche Hinweise zum Überladen von Operatoren findest du hier: \url{http://en.cppreference.com/w/cpp/language/operators}.
}

\subsection{Konstruktor und Destruktor}
Implementiere folgende Klasse.
Füge jedem Konstruktor und Destruktor eine Ausgabe auf der Konsole hinzu, um beim Programmlauf den Lebenszyklus der Objekte nachvollziehen zu können.

\begin{lstlisting}
class Vector3 {
public:
	Vector3();											// initialize vector with zero
	Vector3(double a, double b, double c);		// initialize vector with a, b, c
	Vector3(const Vector3 &other);				// copy constructor: copy a vector
	~Vector3();											// destructor: destroy the vector
private:
	double a, b, c;									// vector components
};
\end{lstlisting}

Der Copy-Konstruktor wird aufgerufen, wenn das Objekt kopiert werden soll, z.B. für eine Call-by-Value Parameterübergabe.
Jeder Copy-Konstruktor benötigt eine Referenz auf ein Objekt vom gleichen Typ wie die Klasse selbst als Parameter. 
Sinnvollerweise wird noch \lstinline{const} vor oder nach der Typbezeichnung eingefügt (aber vor \lstinline{\&}), da typischerweise das Ursprungsobjekt nicht verändert wird.

Der Destruktor wird aufgerufen, sobald die Lebenszeit eines Objekts endet. Er wird verwendet, um Ressourcen die das Objekt besitzt freizugeben.
Die Syntax des Prototypen lautet

\begin{lstlisting}
~ClassName();
\end{lstlisting}

und die Implementation entsprechend

\begin{lstlisting}
ClassName::~ClassName() { /* destructor implementation ... */}
\end{lstlisting}

\hints{
	\item Es dürfen eine beliebige Anzahl an Konstruktoren mit verschiedenen Parametersätzen existieren.
	\item Der Compiler wird automatisch einen \lstinline{public} Destruktor und \lstinline{public} Copy-Konstruktor erzeugen, falls sie nicht \emph{deklariert} wurden. Ebenso wird ein \lstinline{public} Defaultkonstruktor (keine Argumente) automatisch vom Compiler generiert, falls überhaupt keine Konstruktoren deklariert wurden.
    
    Falls sie jedoch \emph{deklariert} wurden, musst du auch eine Implementierung angeben.
	\item Würden beim Copy-Konstruktor \lstinline{other} by-Value übergeben werden, müsste eine Kopie von \lstinline{other} angelegt werden.
	Dazu würde der Copy-Konstruktor aufgerufen, was zu einer unendlichen Rekursion führt, bis der Stack seine maximale Größe überschreitet und das Programm abstürzt.
}

\subsection{Vektoraddition, -subtraktion und Skalarprodukt}
Erweitere die Klasse um folgende \lstinline{public} Funktionen, um Vektoren durch \lstinline{v1 + v2}, \lstinline{v1 - v2} und \lstinline{v1 * v2} addieren/subtrahieren und das Skalarprodukt bilden zu können, indem die Operatoren $+$, $-$ und $*$ überladen werden.

\begin{lstlisting}
Vector3 operator+(Vector3 rhs);		// add two vectors component-by-component
Vector3 operator-(Vector3 rhs);		// subtract two vectors component-by-component
double operator*(Vector3 rhs);		// determine the dot product of two vectors
\end{lstlisting}

Innerhalb der Methode kannst du durch \lstinline{a}, \lstinline{b} und \lstinline{c} auf eigene Attribute und über  \lstinline{rhs.a}, \lstinline{rhs.b} und \lstinline{rhs.c} auf Attribute der rechten Seite zugreifen.
Denke daran, bei der Implementation den Klassen den Scope der Klasse in der Sourcedatei vor jeder Elementbezeichnung durch zwei Doppelpunkte getrennt anzugeben.

\begin{lstlisting}
Vector3 Vector3::operator+(Vector3 rhs) {/* function implementation ...*/}
Vector3 Vector3::operator-(Vector3 rhs) {/* function implementation ...*/}
double Vector3::operator*(Vector3 rhs)  {/* function implementation ...*/}
\end{lstlisting}

\hints{
	\item Der Parameter \lstinline{rhs} steht für die rechte Seite ( \enquote{right-hand-side}) des jeweiligen Operators.
	Dadurch, dass der Operator als Member der Klasse deklariert wurde, nimmt die aktuelle Instanz hierbei automatisch die linke Seite der Operation (\enquote{left-hand side}) an.
	\item Der Rückgabetyp eines Skalarprodukts (dot product) ist kein \lstinline{Vector3} sondern ein Skalar (\lstinline{double})!
}

\subsection{Ausgabe}
Überlade den \lstinline{operator<<} zur Ausgabe eines Vektors mit der gewohnten \lstinline{std::cout $<<$ ...} Syntax, indem du den folgenden Funktionsprototypen \textbf{außerhalb} der Klassendefinition setzt

\begin{lstlisting}
std::ostream& operator<<(std::ostream &out, Vector3 rhs);
\end{lstlisting}

und innerhalb der Sourcedatei wie folgt implementierst.

\begin{lstlisting}
std::ostream& operator<<(std::ostream &out, Vector3 rhs)  {
	out << ... ;
	return out;
}
\end{lstlisting}

Da der \lstinline{operator<<} außerhalb der Klasse \lstinline{Vector3} liegt, hat dieser keinen Zugriff auf die privaten Member der Klasse.

Du hast zwei Möglichkeiten, Zugriff auf diese zu erlangen: per Getter und per \lstinline{friend}-Deklaration.

\paragraph{Getter}
Definiere die folgenden Gettermethoden die die Werte für die \lstinline{private} Attribute \lstinline{a}, \lstinline{b} und \lstinline{c} zurückgeben:
\begin{lstlisting}
  double getA();	// get the first component
  double getB();	// get the second component
  double getC();	// get the third component
\end{lstlisting}

\paragraph{friend}
Füge die folgende Zeile am Ende der Klasse \texttt{Vector} hinzu:
\begin{lstlisting}
    friend  std::ostream& operator<<(std::ostream&, Vector3);
\end{lstlisting}
Von nun an, kann die entsprechende Funktion auf alle privaten Member der Klasse \texttt{Vector} zugreifen, was insbesondere praktisch ist, falls die Klasse verändert werden soll.

\hints{
	\item Denke daran, den Header \lstinline{iostream} einzubinden.
	\item Diesmal musste die Überladung \textbf{außerhalb} der \lstinline{Vektor3}-Klasse definiert werden, weil das \lstinline{Vektor3}-Objekt auf der rechten Seite der Operation steht.
	Als linke Seite wird hierbei ein \lstinline{std::ostream}-Objekt (wie z.B. \lstinline{std::cout}) erwartet, um Ausgabeketten \lstinline{std::cout << ... << ...} zu ermöglichen.
	Hierzu muss das Ausgabeobjekt auch zurückgegeben werden, damit das \lstinline{std::ostream}-Objekt aber nicht jedes Mal kopiert wird, wird es als Referenz \lstinline{&} durchgereicht.
	\item Anstatt Getter und Setter für \lstinline{private} Attribute zu schreiben, kann man auch einer Klasse oder Funktion vollen Zugriff mit Hilfe des Schlüsselworts \lstinline{friend} erlauben. In der nächsten Übung wird hierauf noch einmal eingegangen.
}


\subsection{Testen}
Teste deine bisher definierten Methoden und Funktionen.
Probiere auch Kombinationen von verschiedenen Operatoren aus und beobachte das Ergebnis.
Schreibe auch eine einfache Funktion, die Vektoren als Parameter nimmt.
Wie du siehst, werden sehr viele \lstinline{Vector3}-Objekte erstellt, kopiert und gelöscht.
Dies liegt daran, dass die Objekte immer per Call-By-Value übergeben und dabei kopiert werden.
Wie dies vermieden werden kann, siehst du im nächsten Teil des Praktikums.
