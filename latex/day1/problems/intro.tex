\section*{Einführung}
Für alle Übungen des C/C++ Praktikums wird Eclipse zusammen mit dem C/C++ Development Tooling (CDT) und dem  \emph{GNU project C and C++ compiler} (GCC) verwendet.
Wir gehen davon aus, dass der generelle Umgang mit Eclipse bereits aus der Java-Programmierung bekannt ist.

Die Materialien zur Vorlesung und Übung sind in zwei Git-Repositories zu finden.
Diese beinhalten zum einen die Folien und die Programmbeispiele aus der Vorlesung und zum anderen diese Übungsblätter, Vorlagen für die letzten beiden Tage des Praktikums und alle Lösungen zu den Übungsaufgaben.

\begin{itemize}
	\item \textbf{Vorlesung:} \url{https://github.com/Echtzeitsysteme/tud-cpp-lecture}
	\item \textbf{Übungen/Git- und Eclipseeinführung:} \url{https://github.com/Echtzeitsysteme/tud-cpp-exercises}
\end{itemize}

Weiterhin findest du im Übungsrepository eine kleine Einführung in Git in Eclipse.
Es ist ausdrücklich empfohlen sich sowohl die Unterlagen mit Git herunterzuladen, als auch Git lokal für die eigenen Übungsaufgaben zu verwenden.
Ein Git-Plugin ist in neueren Eclipseversionen standardmäßig verfügbar, das zum Beispiel über \textbf{Window $\rightarrow$ Open Perspective\dots $\rightarrow$ Git} \LK{Eclipse-spezifisch} geöffnet verwendet werden kann.

\hints{
	\item Bei Fragen und Problemen aktiv um Hilfe bitten!
	\item Alle Lösungen enthalten ein Makefile und können entweder über die Kommandozeile mit Hilfe von \texttt{make} kompiliert werden, oder in Eclipse, indem du sie als Makefileprojekt importierst.
	Achte darauf, beim ersten Ausführen innerhalb von Eclipse \texttt{gdb/mi} (ein Interface für den Debugger \texttt{gdb}) auszuwählen.
	\item Folgende Eclipse Shortcuts könnten sich dir im Verlauf des Praktikums als nützlich erweisen:
}

\begin{tabular}{l|l|p{11.5cm}}
    \toprule
    \textbf{Tastenkürzel} & \textbf{Befehl} & \textbf{Beschreibung}\\
    \midrule
	Ctrl+Space & Autocomplete &
	Anzeige von Vervollständigungshinweisen (z.B. nach \texttt{std::} oder \texttt{main})
	\\
	Alt+Shift+R & Rename &
	Umbennenen von Variablen, Funktionen, Klassen, \dots
	\\
	Ctrl+N & New &
	Anlegen neuer Ressourcen (Dateien, Projekte, \dots)
	\\
	Ctrl+Tab & Header$\leftrightarrow$Source &
	Wechsel zwischen der Header- und der Implementierungsdatei
	\\
	Ctrl+Click/F3 & Go to &
	Navigiert zu der Definition eines Elements (Funktion, Klasse, Variable, \dots)
	\\
	Ctrl+B & Build &
	Startet den Buildprozess (Aufruf von Compiler und Linker)
    \\\bottomrule
\end{tabular}
