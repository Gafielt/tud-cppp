\providecommand{\additionalOptionsForClass}{}
\documentclass[
  accentcolor=tud1c,	% Color theme for TUD corporate design
  colorbacktitle,		% Titlepage has colored background for title area
  inverttitle,			% Font color of title on titlepage is inverted
  \additionalOptionsForClass
  %german%,
  %twoside
]{tudexercise}

\parindent1em
%\parskip2ex

\usepackage[ngerman]{babel}
\usepackage[utf8]{inputenc}
\usepackage{listings}
\usepackage{booktabs}
\usepackage{amsmath}
\usepackage{algorithm2e}
\usepackage{hyperref}
\usepackage{xspace}
\usepackage{tabularx}
\usepackage{tikz}
\usepackage{cleveref}
\usepackage{numprint}
\usepackage{paralist}
\usepackage{verbatim}
\usepackage{tocloft} % for manipulating the table of contents

\usetikzlibrary{shapes}
\usetikzlibrary{calc}
\usetikzlibrary{arrows}
\usetikzlibrary{decorations}

\usepackage{pifont}
\newcommand{\cmark}{\ding{51}\xspace}%
\newcommand{\xmark}{\ding{55}\xspace}%

\usepackage{todonotes}
%\usepackage[disable]{todonotes} % Use this line to hide all todos

\definecolor{commentgreen}{RGB}{50,127,50}
\lstloadlanguages{C++,[gnu]make}
\lstset{language=C++}
\lstset{captionpos=b}
\lstset{tabsize=3}
\lstset{breaklines=true}
\lstset{basicstyle=\ttfamily}
\lstset{columns=flexible}
\lstset{keywordstyle=\color{purple}}
\lstset{stringstyle=\color{blue}}
\lstset{commentstyle=\color{commentgreen}}
\lstset{otherkeywords=\#include}
\lstset{showstringspaces=false}
\lstset{keepspaces=true}
\lstset{xleftmargin=1cm}
\lstset{literate=%
	{Ö}{{\"O}}1
	{Ä}{{\"A}}1
	{Ü}{{\"U}}1
	{ß}{{\ss}}2
	{ü}{{\"u}}1
	{ä}{{\"a}}1
	{ö}{{\"o}}1
	{'}{{\textquotesingle}}1
}

\lstnewenvironment{lstmake} %
{\lstset{language=[gnu]make}} %
{}


\newcommand{\superscript}[1]{\ensuremath{^{\textrm{#1}}}}
\newcommand{\subscript}[1]{\ensuremath{_{\textrm{#1}}}}

\newcommand{\setHeader}[1]{
\providecommand{\examheadertitle}{TODO: Titel einbinden}
\renewcommand{\examheadertitle}{#1}
\begin{examheader}
    \examheadertitle
\end{examheader}
}

\newcommand{\hints}[1]{
\paragraph*{Hinweise}
	\begin{itemize}
		\setlength{\itemsep}{0pt}
		#1
	\end{itemize}
}

\newcommand{\optional}{\xspace(optional)}
\newcommand{\experimental}{\xspace(experimentell)}

\usepackage{fancybox}
\newcommand{\optionaltextbox}{
	\bigskip
	\begin{center}
		\ovalbox{\parbox{0.98\textwidth}{Die Klausur kann ohne diese Aufgabe bestanden werden. Wir empfehlen aber sie trotzdem zu bearbeiten.}}
	\end{center}
}
\newcommand{\experimentaltextbox}{
	\bigskip
	\begin{center}
		\ovalbox{\parbox{0.98\textwidth}{Diese Aufgabe wurde neu erstellt und kann noch Fehler und Inkonsistenzen enthalten. Falls euch etwas derartiges auffällt sprecht uns bitte darauf an oder stellt es auf GitHub in den Issuetracker unter \url{https://github.com/Echtzeitsysteme/tud-cpp-exercises/issues}}}
	\end{center}
}

\newcommand{\enquote}[1]{\glqq#1\grqq\xspace}
\newcommand{\filename}[1]{\texttt{#1}}

\newcommand{\RK}[1]{\todo[]{\textbf{RK:} #1}}
\newcommand{\RKi}[1]{\todo[inline]{\textbf{RK:} #1}}

\newcommand{\ExercisePrefix}[1]{$[$#1$]$ \xspace}
\newcommand{\ExercisePrefixBasics}{\ExercisePrefix{G}}
\newcommand{\ExercisePrefixMemory}{\ExercisePrefix{S}}
\newcommand{\ExercisePrefixObjectOrientation}{\ExercisePrefix{O}}
\newcommand{\ExercisePrefixAdvanced}{\ExercisePrefix{F}}
\newcommand{\ExercisePrefixEmbeddedC}{\ExercisePrefix{C}}
\newcommand{\ExercisePrefixElevator}{\ExercisePrefix{A}}
\newcommand{\ExercisePrefixAdditionalInformation}{\ExercisePrefix{Z}}


\newcommand{\exday}{1}

\cppSetTitle

\begin{document}

\cppSetHeaderAndMakeTitle



\section*{Einführung}

C++ --> aktiv Hilfe ersuchen

Eclipse Tutorial

Git Tutorial

Übungsmaterial Link

Einführung in C++

\subsection*{Eclipse Shortcuts}
Das Arbeiten mit Eclipse macht erst so richtig Spaß, wenn man einige Shortcuts beherrscht.
Dazu zählen:
\begin{description}
	\item[Autocomplete (\emph{Ctrl+Space})]
	Hier erhältst du Vervollständigungsvorschläge und andere Hinweise.
	Versuche es mal mit \emph{std::}, \emph{using namespace}.
	Wenn du \emph{main} tippst und \emph{Ctrl+Space} drückst, generiert Eclipse für dich die \emph{main}-Funktion.
	\item[Umbenennen (\emph{Alt+Shift+R})]
	Mit dieser Funktion kannst du Variablen, Funktionen, Klassen,\dots umbenennen.
	\item[Neu (\emph{Ctrl+N})]
	Hier kannst du schnell neue Resourcen (Dateien, Projekte,\dots) anlegen.
	\item[Header$\leftrightarrow$Source (\emph{Ctrl+Tab})]
	Auf diese Weise kannst du schnell zwischen dem Header und der Implementierungdatei hin- und herspringen.
	\item[Go to (\emph{Ctrl+Click}/\emph{F3})]
	Wenn du \emph{Ctrl} gedrückt hältst und auf ein Element klickst (Funktion, Klasse,\dots), navigiert Eclipse dorthin.
	\item[Build (\emph{Ctrl+B})]
	Startet den Buildprozess (Aufruf von Compiler und Linker)
\end{description}



\section{C++ Grundlagen, Funktionen und Strukturierung}

Nach dieser kurzen Einführung darfst du dir jetzt selber die Hände schmutzig machen.

Wir lernen als zunächst die Kontrollstrukturen (if/for/while) von C++ kennen, die weitgehend gleich sind wie in Java.
In den folgenden Aufgaben werden die Unterschiede und Gemeinsamkeiten stückweise erläutert.

\begin{enumerate}

\item Lege ein neues \glqq{}Hello World\grqq{}-Projekt an (\emph{File} $\rightarrow$ \emph{New} $\rightarrow$ \emph{C++ Project} $\rightarrow$ \emph{Hello World C++ Project}).
Kompiliere und starte das Programm.

Jetzt haben wir einen Ausgangspunkt für eine eigene Implementierung.


\item 
Schreibe eine Funktion \emph{printStars(int n)}\footnote{Was die Benennung von Funktionen, Variablen und Klassen angeht, bist du frei.
Für Klassen ist CamelCase wie in Java üblich. Bei Funktionen und Variablen wird zumeist entweder auch Camel Case oder Kleinschreibung mit Unterstrichen verwendet.}, die \emph{n}-mal ein \textbf{*} auf der Konsole ausgibt und mit einem Zeilenvorschub abschließt.
Ein Aufruf von \emph{printStars(5)} sollte folgende Ausgabe generieren:
\begin{lstlisting}
*****
\end{lstlisting}

Platziere die Funktion \emph{vor} der \emph{main}, da sie sonst von dort aus nicht aufgerufen werden kann.
Benutze die erstellte Funktion \emph{printStars(int n)}, um eine weitere Funktion zu schreiben, die eine Figur wie unten dargestellt ausgibt.
Verwende hierzu Schleifen.
\begin{lstlisting}
*****
****
***
**
*
**
***
****
*****
\end{lstlisting}

\item
Lagere dein Programm aus Aufgabenteil b) in eine eigene Datei aus.
Gehe dazu folgendermaßen vor: 

Erstelle eine neue Header-Datei \emph{functions.h} und eine neue Sourcecode-Datei \emph{functions.cpp}. 

Binde \emph{functions.h} in beide Sourcecode-Dateien mittels 
einer \textbf{\#include} ein.
Vergiss nicht, auch in \emph{functions.cpp} den \emph{std}-Namespace sowie \emph{iostream} einzubinden, falls du dort Elemente der Standardbibliothek benutzen möchtest.

Vermeide bei größeren Programmen bitte unbedingt, den Namespace bereits im Header einzubinden!
Dadurch würden alle Dateien, die den Header einbinden, den Namespace ebenfalls einbinden, was zu Namenskonflikten und anderen hässlichen Überraschungen führen kann.
Benutze daher im Header immer voll qualifizierte Bezeichner (z.B. \emph{std::string}, \emph{std::ostream}).

Schreibe nun in \emph{functions.h} \textbf{Funktionsprototypen} für die beiden Funktionen aus b).
Funktionsprototypen dienen dazu, dem Compiler mitzuteilen, dass eine Funktion mit bestimmtem Namen, Parametern und Rückgabewert existiert.
Ein Prototyp ist im wesentlichen eine mit \textbf{;} abgeschlossene Signatur der Funktion ohne Funktionsrumpf.
Der Prototyp von \emph{printStars(int n)} lautet \emph{void printStars(int n);}

Verschiebe deine beiden Funktionen nach \emph{functions.cpp}.
Fertig -- die Ausgabe des Programms sollte sich nicht verändert haben.

\item
Erweitere das in Aufgabenteil c) erstellte Programm um eine Eingabeaufforderung zur Bestimmung der Breite der auszugebenden Figur.
Die Breite soll dabei eine im Programmcode vorgegebene Grenze nicht überschreiten dürfen.
Gib gegebenenfalls eine Fehlermeldung aus.
Zum Einlesen verwendest du \texttt{cin >{}> \emph{variable}}.
Erstelle auch für diesen Aufgabenteil eine eigene Funktion und lagere diese nach \emph{functions.cpp} aus.

\item Statt eines einzelnen Zeichens soll nun das fortlaufende Alphabet ausgegeben werden.
Sobald das Ende des Alphabets erreicht wurde, beginnt die Ausgabe erneut bei \emph{a}. Beispiel:
\begin{lstlisting}
abc
de
f
gh
ijk
\end{lstlisting}

Implementiere dazu eine Funktion \emph{char nextChar()}.
Diese soll bei jedem Aufruf das nächste auszugebende Zeichen von Typ \emph{char} zurückgeben, beginnend bei \emph{\textquotesingle a\textquotesingle}.
Dazu muss sich \emph{nextChar()} intern das aktuelle Zeichen merken.
Dies kann durch die Verwendung von \emph{static}-Variablen erreicht werden. \emph{static}-Variablen sind Variablen, die ihren alten Wert beim Wiedereintritt in die Funktion annehmen.
Ein statische Variable \emph{c} wird mittels 
\begin{lstlisting}
static char c = 'a';
\end{lstlisting}
deklariert.
In diesem Fall wird c \textbf{einmalig zu Beginn des Programms} mit 'a' initialisiert und kann später beliebig verändert werden.
\end{enumerate} 



\end{document}
