\section{Template Funktionen}
\subsection{Templatefunktionen implementieren}
Implementiere die folgende Funktion, die das Maximum von zwei Variablen liefert:

\begin{lstlisting}
template<class T>
const T &maximum(const T &t1, const T &t2);
\end{lstlisting}

Durch die Verwendung von Templates soll die Funktion mit verschiedenen Datentypen funktionieren.
Teste deine Implementation.

In der Vorlesung haben wir gesehen, dass jede Verwendung von \texttt{t1} und \texttt{t2} in \texttt{maximum} eine Schnittstelle induziert, die der Typ \texttt{T} bereitstellen muss.
Das bedeutet, dass \texttt{T} alle Konstruktoren, Methoden und Operatoren zur Verfügung stellen muss, die in \texttt{maximum} genutzt werden.

Wie sieht diese Schnittstelle in diesem Fall aus?

\hints{
    \item Es macht keinen Unterschied, ob du \texttt{class} oder \texttt{typename} in der Template-Deklaration nutzt.    
}

\subsection{Explizite Angabe der Typparameter}
Lege nun zwei Variablen vom Typ \texttt{int} und \texttt{short} an, und versuche, mittels \texttt{maximum()} das Maximum zu bestimmen.
Der Compiler wird mit der Fehlermeldung \textbf{no matching function for call...} abbrechen, da er nicht weiß, ob \texttt{int} oder \texttt{short} der Template-Parameter sein soll.
Gib deshalb den Template-Parameter mittels \texttt{maximum<int>()} beim Aufruf von \texttt{maximum()} explizit an.
Die übergebenen Parameter werden dabei vom Compiler automatisch in den gewünschten Typ umgewandelt.

\subsection{Induzierte Schnittstelle implementieren}
Erstelle eine Klasse \texttt{C}, die eine Zahl als Attribut beinhaltet. Implementiere einen passenden Konstruktor sowie einen Getter für diese Zahl. Nun wollen wir unsere Funktion  \texttt{maximum()} verwenden, um zu entscheiden, welches von zwei \texttt{C}-Objekten die größere Zahl beinhaltet.
Überlege dir, was zu tun ist, und implementiere es.

\hints{
	\item Die Klasse \texttt{C} muss mindestens die durch \texttt{maximum} induzierte Schnittstelle implementieren.
}
