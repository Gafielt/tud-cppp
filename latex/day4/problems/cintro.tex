\section{C Einführung \optional}

\optionaltextbox
In den nächsten Tagen werden wir Programme für eine Embedded-Plattform in C entwickeln.
Da C++ aus C entstand, sind viele Features von C++ nicht in C enthalten. Im Folgenden sollen die Hauptunterschiede verdeutlicht werden.

\begin{itemize}
	\item Kein OO-Konzept, keine Klassen, nur Strukturen (\lstinline{struct}).
	\item Keine Templates
	\item Keine Referenzen, nur Zeiger und Werte
	\item Kein \lstinline{new} und \lstinline{delete}, sondern \lstinline{malloc()} und \lstinline{free()} (\verb|#include <stdlib.h>|)
	\item Je nach Sprachstandard müssen Variablen am Anfang der Funktion deklariert werden (Standard-Versionen $<$ C99)
	\item Parameterlose Funktionen müssen \lstinline{void} als Parametertyp haben, leere Klammern () bedeuten, dass beliebige Argumente erlaubt sind.
	\item Keine Streams, stattdessen \lstinline{(f)printf} zur Ausgabe auf Konsole und in Dateien (\verb|#include <stdio.h>|)
	\item Kein \lstinline{bool} Datentyp, stattdessen wird 0 als \lstinline{false} und alle anderen Zahlen als \lstinline{true} gewertet
	\item Keine Default-Argumente
	\item Keine \lstinline{std::string} Klasse, nur \lstinline{char}-Arrays, die mit dem Nullbyte (\verb|'\0'|) abgeschlossen werden.
	\item Keine Namespaces
\end{itemize}

Da einige dieser Punkte sehr entscheidend sind, werden wir auf diese im Detail eingehen.

\subsection{Kein OO-Konzept}
In C gibt es keine Klassen, weshalb die Programmierung in C eher Pascal statt C++ ähnelt.
Stattdessen gibt es Strukturen (\lstinline{struct}), die mehrere Variablen zu einem Datentyp zusammenfassen, was vergleichbar mit Records in Pascal oder -- allgemein -- mit Klassen ohne Methoden und ohne Vererbung ist.

Die Syntax dafür lautet

\lstinputlisting{problems/listings/cintro_structs.c}

Zum Beispiel

\lstinputlisting{problems/listings/cintro_structs_ex.c}

Die Sichtbarkeit aller Attribute ist automatisch \lstinline{public}.

Um den definierten \textbf{struct} als Datentyp zu verwenden, muss man zusätzlich zum Namen das Schlüsselwort \lstinline{struct} angeben:

\lstinputlisting{problems/listings/cintro_structs_func.c}

Um den zusätzlichen Schreibaufwand zu vermeiden, wird in der Praxis oft ein \lstinline{typedef} auf den \lstinline{struct} definiert:

\lstinputlisting{problems/listings/cintro_structs_typedef.c}

Man kann die Deklaration eines \lstinline{struct} auch direkt in den \lstinline{typedef} einbauen:

\lstinputlisting{problems/listings/cintro_structs_short.c}

\subsection{Kein \lstinline{new} und \lstinline{delete}}

Anstelle von \lstinline{new} und \lstinline{delete} werden die Funktionen \lstinline{malloc} und \lstinline{free} verwendet, um Speicher auf dem Heap zu reservieren.
Diese sind im Header \filename{stdlib.h} deklariert.

\lstinputlisting{problems/listings/cintro_mallocfree.c}

\subsection{Ausgabe auf Konsole per \lstinline{printf}}

Um Daten auf der Konsole auszugeben, kann die Funktion \lstinline{printf} verwendet werden.
\lstinline{printf()} nimmt einen Format-String sowie eine beliebige Anzahl weiterer Argumente entgegen.
Der Format-String legt fest, wie die nachfolgenden Argumente ausgegeben werden.
Mittels \textbackslash n kann man einen Zeilenvorschub erzeugen. Um \lstinline{printf()} zu nutzen, muss der Header \filename{stdio.h} eingebunden werden.

\lstinputlisting{problems/listings/cintro_printf.c}

Weitere Möglichkeiten von \lstinline{printf()} findest du unter \url{http://www.cplusplus.com/reference/cstdio/printf/}.

\subsection{}
Schreibe ein C-Programm, welches alle geraden Zahlen von 0 bis 200 formatiert ausgibt.
Die Formatierung soll entsprechend dem Beispiel erfolgen:

\lstinputlisting{problems/listings/cintro_format.c}

\subsection{}
Versuche, beliebige (einfache) Programme der vergangenen Tage in reinem C auszudrücken (Schwierigkeitsgrad sehr unterschiedlich!).
