\section{C Einführung}
In den nächsten Tagen werden wir Programme für eine Embedded-Plattform in C entwickeln.
Da C++ aus C entstand, sind viele Features von C++ nicht in C enthalten. Im Folgenden sollen die Hauptunterschiede verdeutlicht werden.

\begin{itemize}
	\item Kein OO-Konzept, keine Klassen, nur Strukturen (\texttt{struct}).
	\item Keine Templates
	\item Keine Referenzen, nur Zeiger und Werte
	\item Kein \texttt{new} und \texttt{delete}, sondern \texttt{malloc()} und \texttt{free()} (\verb|#include <stdlib.h>|)
	\item Je nach Sprachstandard müssen Variablen am Anfang der Funktion deklariert werden (Standard-Versionen $<$ C99)
	\item Parameterlose Funktionen müssen \texttt{void} als Parametertyp haben, leere Klammern () bedeuten, dass beliebige Argumente erlaubt sind.
	\item Keine Streams, stattdessen \texttt{(f)printf} zur Ausgabe auf Konsole und in Dateien (\verb|#include <stdio.h>|)
	\item Kein \texttt{bool} Datentyp, stattdessen wird 0 als \texttt{false} und alle anderen Zahlen als \texttt{true} gewertet
	\item Keine Default-Argumente
	\item Keine \texttt{std::string} Klasse, nur \texttt{char}-Arrays, die mit dem Nullbyte (\verb|'\0'|) abgeschlossen werden.
	\item Keine Namespaces
\end{itemize}

Da einige dieser Punkte sehr entscheidend sind, werden wir auf diese im Detail eingehen.

\subsection{Kein OO-Konzept}
In C gibt es keine Klassen, weshalb die Programmierung in C eher Pascal statt C++ ähnelt.
Stattdessen gibt es Strukturen (\texttt{struct}), die mehrere Variablen zu einem Datentyp zusammenfassen, was vergleichbar mit Records in Pascal oder -- allgemein -- mit Klassen ohne Methoden und ohne Vererbung ist.

Die Syntax dafür lautet

\begin{lstlisting}
struct MyStruct {
	<Type1> <Name11>, <Name12>, ...;
	<Type2> <Name21>, <Name22>, ...;
};
\end{lstlisting}

Zum Beispiel

\begin{lstlisting}
struct Point {
	int x;
	int y;
};
\end{lstlisting}

Die Sichtbarkeit aller Attribute ist automatisch \texttt{public}.

Um den definierten \textbf{struct} als Datentyp zu verwenden, muss man zusätzlich zum Namen das Schlüsselwort \texttt{struct} angeben:

\begin{lstlisting}
void foo(struct Point *p) {
...
}

int main(void) {
	struct Point point;
	foo(&point);
}
\end{lstlisting}

Um den zusätzlichen Schreibaufwand zu vermeiden, wird in der Praxis oft ein \texttt{typedef} auf den \texttt{struct} definiert:

\begin{lstlisting}
typedef struct Point Point_t;
Point_t point;
\end{lstlisting}

Man kann die Deklaration eines \texttt{struct} auch direkt in den \texttt{typedef} einbauen:

\begin{lstlisting}
typedef struct {
	int x;
	int y;
} Point;
\end{lstlisting}

\subsection{Kein \texttt{new} und \texttt{delete}}

Anstelle von \texttt{new} und \texttt{delete} werden die Funktionen \texttt{malloc} und \texttt{free} verwendet, um Speicher auf dem Heap zu reservieren.
Diese sind im Header \texttt{stdlib.h} deklariert.

\begin{lstlisting}
#include <stdlib.h>
Point *points = malloc(10 * sizeof(Point)); // reserve memory for 10 points
// ...
free(points);
\end{lstlisting}

\subsection{Ausgabe auf Konsole per \texttt{printf}}

Um Daten auf der Konsole auszugeben, kann die Funktion \texttt{printf} verwendet werden.
\texttt{printf()} nimmt einen Format-String sowie eine beliebige Anzahl weiterer Argumente entgegen.
Der Format-String legt fest, wie die nachfolgenden Argumente ausgegeben werden.
Mittels \textbackslash n kann man einen Zeilenvorschub erzeugen. Um \texttt{printf()} zu nutzen, muss der Header \texttt{stdio.h} eingebunden werden.

\begin{lstlisting}
#include <stdio.h>
printf("Hallo Welt\n"); // Hallo Welt + neue Zeile ausgeben

int i;
printf("i = %d\n", i);  // Integer ausgeben
printf("i = %3d\n", i); // Integer ausgeben, auf drei Stellen auffüllen

int i;
char c;
printf("c = %c, i = %d\n", c, i); // Zeichen und Integer ausgeben
\end{lstlisting}

Weitere Möglichkeiten von \texttt{printf()} findest du unter \url{http://www.cplusplus.com/reference/cstdio/printf/}.

\subsection{}
Schreibe ein C-Programm, welches alle geraden Zahlen von 0 bis 200 formatiert ausgibt.
Die Formatierung soll entsprechend dem Beispiel erfolgen:

\begin{lstlisting}
2   4   6   8  10
12  14  16  ...
\end{lstlisting}

\subsection{}
Versuche, beliebige (einfache) Programme der vergangenen Tage in reinem C auszudrücken (Schwierigkeitsgrad sehr unterschiedlich!).
