%!TeX spellcheck = de_DE

\input{00_common/common_head.tex}
\title{Übung zum\linebreak[1]C/C++-Praktikum \linebreak[1] Fachgebiet Echtzeitsysteme\linebreak[1]\linebreak[1] Grundlagen}

\setcounter{section}{0}

%optional parameters to speed up build (increases pdf file size)
%\pdfcompresslevel=0
%\pdfobjcompresslevel=0
\begin{document}

\maketitle

\noindent Die Aufgaben für das C/C++-Praktikum sind thematisch sortiert.
Zu Beginn jedes Themengebiets der Vortragsfolien ist vermerkt, welche Aufgaben zu diesem Themengebiet gehören (%
\ExercisePrefixBasics: Grundlagen; 
\ExercisePrefixMemory: Speicherverwaltung;
\ExercisePrefixObjectOrientation: Objektorientierung;
\ExercisePrefixAdvanced: Fortgeschrittene Themen;
\ExercisePrefixEmbeddedC: (Embedded) C;
\ExercisePrefixElevator: Optionale Zusatzaufgaben (C++)).
%
\setcounter{tocdepth}{1}
\setlength\cftsecnumwidth{10em}
\setlength\cftbeforesecskip{.1em} % line skip between sections ("sec")

\setHeader{Aufgaben zu C/C++-Grundlagen}
%!TeX spellcheck = de_DE

\section*{Einführung}
Für alle Übungen des C/C++-Praktikums wird CodeLite als IDE verwendet. Als Compiler kommt Clang zum Einsatz.

Die Materialien zur Vorlesung und Übung sind in einem GitHub-Repository zu finden: \url{https://github.com/Echtzeitsysteme/tud-cppp}.
Dieses beinhaltet zum einen die Folien und die Programmbeispiele aus der Vorlesung und zum anderen diese Übungsblätter, Vorlagen für die letzten beiden Tage des Praktikums und alle Lösungen zu den Übungsaufgaben.

\hints{
\item Bei Fragen und Problemen bitte aktiv um Hilfe!
\item Alle Lösungen zum C++-Teil enthalten ein Makefile und können entweder über die
Kommandozeile mit Hilfe von \lstinline{make} kompiliert werden, oder in CodeLite, indem du sie als Projekt importierst.
\item Folgende Tastenkürzel könnten sich dir im Verlauf des Praktikums als nützlich erweisen:
}


\begin{tabular}{l|l|p{11.5cm}}
    \toprule
    \textbf{Tastenkürzel} & \textbf{Befehl} & \textbf{Beschreibung}\\
    \midrule
	Ctrl+Space & Autocomplete &
	Anzeige von Vervollständigungshinweisen (z.B. nach \texttt{std::} oder \texttt{main})
	\\
	Alt+Shift+L & Rename local &
	Umbennenen von lokalen Variablen
	\\
	Alt+Shift+H & Rename symbol &
	Umbennenen von Funktionen und Klassen
	\\
	Ctrl+N & New &
	Anlegen neuer Datei
	\\
	F12 & Switch tab &
	Wechsel zwischen der Header- und der Implementierungsdatei
	\\
	F7 & Build &
	Startet den Buildprozess (Aufruf von Compiler und Linker)
    \\\bottomrule
\end{tabular}


\section*{Projekte in CodeLite importieren}

Die angebotenen Musterlösungs-/Microcontroller-Projekte basieren auf Makefiles.
Daher ist es wichtig, dass du sie entsprechend importierst: \menuPath{Workspace \menuSep Add an existing project} und dann zur entsprechenden \filename{.project}-Datei navigieren.


\newpage
\input{01_basics/problems/helloWorld}
\newpage
\input{01_basics/problems/basics}
\newpage
\input{01_basics/problems/classes}
\newpage
\input{01_basics/problems/overloading}

\cclicense

\end{document}
