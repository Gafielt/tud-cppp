%!TeX spellcheck = de_DE

\providecommand{\additionalOptionsForClass}{}
\documentclass[
  accentcolor=tud1c,	% Color theme for TUD corporate design
  colorbacktitle,		% Titlepage has colored background for title area
  inverttitle,			% Font color of title on titlepage is inverted
  \additionalOptionsForClass
  %german%,
  %twoside
]{tudexercise}

\parindent1em
%\parskip2ex

\usepackage[ngerman]{babel}
\usepackage[utf8]{inputenc}
\usepackage{listings}
\usepackage{booktabs}
\usepackage{amsmath}
\usepackage{algorithm2e}
\usepackage{hyperref}
\usepackage{xspace}
\usepackage{tabularx}
\usepackage{tikz}
\usepackage{cleveref}
\usepackage{numprint}
\usepackage{paralist}
\usepackage{verbatim}
\usepackage{tocloft} % for manipulating the table of contents

\usetikzlibrary{shapes}
\usetikzlibrary{calc}
\usetikzlibrary{arrows}
\usetikzlibrary{decorations}

\usepackage{pifont}
\newcommand{\cmark}{\ding{51}\xspace}%
\newcommand{\xmark}{\ding{55}\xspace}%

\usepackage{todonotes}
%\usepackage[disable]{todonotes} % Use this line to hide all todos

\definecolor{commentgreen}{RGB}{50,127,50}
\lstloadlanguages{C++,[gnu]make}
\lstset{language=C++}
\lstset{captionpos=b}
\lstset{tabsize=3}
\lstset{breaklines=true}
\lstset{basicstyle=\ttfamily}
\lstset{columns=flexible}
\lstset{keywordstyle=\color{purple}}
\lstset{stringstyle=\color{blue}}
\lstset{commentstyle=\color{commentgreen}}
\lstset{otherkeywords=\#include}
\lstset{showstringspaces=false}
\lstset{keepspaces=true}
\lstset{xleftmargin=1cm}
\lstset{literate=%
	{Ö}{{\"O}}1
	{Ä}{{\"A}}1
	{Ü}{{\"U}}1
	{ß}{{\ss}}2
	{ü}{{\"u}}1
	{ä}{{\"a}}1
	{ö}{{\"o}}1
	{'}{{\textquotesingle}}1
}

\lstnewenvironment{lstmake} %
{\lstset{language=[gnu]make}} %
{}


\newcommand{\superscript}[1]{\ensuremath{^{\textrm{#1}}}}
\newcommand{\subscript}[1]{\ensuremath{_{\textrm{#1}}}}

\newcommand{\setHeader}[1]{
\providecommand{\examheadertitle}{TODO: Titel einbinden}
\renewcommand{\examheadertitle}{#1}
\begin{examheader}
    \examheadertitle
\end{examheader}
}

\newcommand{\hints}[1]{
\paragraph*{Hinweise}
	\begin{itemize}
		\setlength{\itemsep}{0pt}
		#1
	\end{itemize}
}

\newcommand{\optional}{\xspace(optional)}
\newcommand{\experimental}{\xspace(experimentell)}

\usepackage{fancybox}
\newcommand{\optionaltextbox}{
	\bigskip
	\begin{center}
		\ovalbox{\parbox{0.98\textwidth}{Die Klausur kann ohne diese Aufgabe bestanden werden. Wir empfehlen aber sie trotzdem zu bearbeiten.}}
	\end{center}
}
\newcommand{\experimentaltextbox}{
	\bigskip
	\begin{center}
		\ovalbox{\parbox{0.98\textwidth}{Diese Aufgabe wurde neu erstellt und kann noch Fehler und Inkonsistenzen enthalten. Falls euch etwas derartiges auffällt sprecht uns bitte darauf an oder stellt es auf GitHub in den Issuetracker unter \url{https://github.com/Echtzeitsysteme/tud-cpp-exercises/issues}}}
	\end{center}
}

\newcommand{\enquote}[1]{\glqq#1\grqq\xspace}
\newcommand{\filename}[1]{\texttt{#1}}

\newcommand{\RK}[1]{\todo[]{\textbf{RK:} #1}}
\newcommand{\RKi}[1]{\todo[inline]{\textbf{RK:} #1}}

\newcommand{\ExercisePrefix}[1]{$[$#1$]$ \xspace}
\newcommand{\ExercisePrefixBasics}{\ExercisePrefix{G}}
\newcommand{\ExercisePrefixMemory}{\ExercisePrefix{S}}
\newcommand{\ExercisePrefixObjectOrientation}{\ExercisePrefix{O}}
\newcommand{\ExercisePrefixAdvanced}{\ExercisePrefix{F}}
\newcommand{\ExercisePrefixEmbeddedC}{\ExercisePrefix{C}}
\newcommand{\ExercisePrefixElevator}{\ExercisePrefix{A}}
\newcommand{\ExercisePrefixAdditionalInformation}{\ExercisePrefix{Z}}

\title{Übung zum\linebreak[1]C/C++-Praktikum \linebreak[1] Fachgebiet Echtzeitsysteme\linebreak[1]\linebreak[1] Grundlagen}

\setcounter{section}{0}

%optional parameters to speed up build (increases pdf file size)
%\pdfcompresslevel=0
%\pdfobjcompresslevel=0
\begin{document}

\maketitle

\noindent Die Aufgaben für das C/C++-Praktikum sind thematisch sortiert.
Zu Beginn jedes Themengebiets der Vortragsfolien ist vermerkt, welche Aufgaben zu diesem Themengebiet gehören (%
\ExercisePrefixBasics: Grundlagen; 
\ExercisePrefixMemory: Speicherverwaltung;
\ExercisePrefixObjectOrientation: Objektorientierung;
\ExercisePrefixAdvanced: Fortgeschrittene Themen;
\ExercisePrefixEmbeddedC: (Embedded) C;
\ExercisePrefixElevator: Optionale Zusatzaufgaben (C++)).
%
\setcounter{tocdepth}{1}
\setlength\cftsecnumwidth{10em}
\setlength\cftbeforesecskip{.1em} % line skip between sections ("sec")

\setHeader{Aufgaben zu C/C++-Grundlagen}
\section*{Einführung}
Für alle Übungen des C/C++ Praktikums wird CodeLite als IDE verwendet. Als Compiler kommt Clang zum Einsatz.
Eine Einführung in den Umgang mit CodeLite findet am ersten Praktikumstag statt.

Die Materialien zur Vorlesung und Übung sind in zwei Git-Repositories zu finden.
Diese beinhalten zum einen die Folien und die Programmbeispiele aus der Vorlesung und zum anderen diese Übungsblätter, Vorlagen für die letzten beiden Tage des Praktikums und alle Lösungen zu den Übungsaufgaben.

\begin{itemize}
	\item \textbf{Vorlesung:} \url{https://github.com/Echtzeitsysteme/tud-cpp-lecture}
	\item \textbf{Übungen/Git- und CodeLite-Einführung:} \url{https://github.com/Echtzeitsysteme/tud-cpp-exercises}
\end{itemize}

Falls du mit dem Übungsblatt für den dafür vorgesehenen Tag innerhalb der Präsenzzeit fertig werden solltest kannst du außerdem gerne mit dem nächsten Übungsblatt anfangen.

\hints{
	\item Bei Fragen und Problemen aktiv um Hilfe bitten!
	\item Alle Lösungen enthalten ein Makefile und können entweder über die
Kommandozeile mit Hilfe von \lstinline{make} kompiliert werden, oder in CodeLite, indem du sie als Projekt importierst.
	\item Folgende Shortcuts könnten sich dir im Verlauf des Praktikums als nützlich erweisen:
}


\begin{tabular}{l|l|p{11.5cm}}
    \toprule
    \textbf{Tastenkürzel} & \textbf{Befehl} & \textbf{Beschreibung}\\
    \midrule
	Ctrl+Space & Autocomplete &
	Anzeige von Vervollständigungshinweisen (z.B. nach \texttt{std::} oder \texttt{main})
	\\
	Alt+Shift+L & Rename local &
	Umbennenen von lokalen Variablen
	\\
	Alt+Shift+H & Rename symbol &
	Umbennenen von Funktionen und Klassen
	\\
	Ctrl+N & New &
	Anlegen neuer Datei
	\\
	F12 & Switch tab &
	Wechsel zwischen der Header- und der Implementierungsdatei
	\\
	F7 & Build &
	Startet den Buildprozess (Aufruf von Compiler und Linker)
    \\\bottomrule
\end{tabular}


\section*{Musterlösungs-/Microcontroller-Projekte in CodeLite importieren}

Die angebotenen Musterlösungs-/Microcontroller-Projekte basieren auf Makefiles.
Daher ist es wichtig, dass du sie entsprechend importierst: \textbf{Workspace -> Add an existing project} und dann zur entsprechenden \filename{.project}-Datei navigieren.


\newpage
\section{Hello World}
Lege ein neues C++ Projekt an, indem du \textbf{Workspace $\rightarrow$ New project} im CodeLite Menü wählst und als Projekttyp \textbf{CPPP/C++ Projekt} auswählst.
Wähle einen passenden Projektnamen, z.b. \texttt{Tag1\_Aufgabe1}. Als Projektpfad sollte \texttt{/home/cppp/CPPP/Workspace} ausgewählt sein. Achte darauf, dass der Haken bei \texttt{Create the project under a separate directory} gesetzt ist.

Das Projekt enthält bereits eine Datei \texttt{src/main.cpp}. Erweitere die \texttt{main}-Funktion wie gezeigt und kompiliere das Projekt mit einem Klick auf das Build-Symbol (grünes Rechteck mit weißem Pfeil). Führe dann das Programm mit einem Klick auf das Zahnrad-Symbol aus.

% fix that avoids undesired page breaks
\begin{minipage}{\textwidth} 
	\lstinputlisting{01_basics/problems/listings/helloworld.cpp}
\end{minipage}

Jedes vollständige C++ Programm muss \textbf{genau eine} Funktion mit Namen \lstinline{main} und Rückgabetypen \lstinline{int} außerhalb von Klassen im globalen Namensraum besitzen.
Andernfalls wird der Linker mit der Fehlermeldung \emph{undefined reference to 'main'} abbrechen.
Der Rückgabetyp wird verwendet um dem Aufrufer (Betriebssystem, Shell, \dots) den Erfolg oder Misserfolg der Ausführung zu signalisieren.
Typischerweise wird im Erfolgsfall 0 zurückgegeben.

Die erste Zeile des obigen Programms bindet den Header der \lstinline{iostream} Bibliothek ein, welche unter anderem Klassen und Funktionen zur Ein- und Ausgabe mit Hilfe von $<<$ (\emph{insertion operator}) und $>>$ (\emph{extraction operator}) anbietet.
Diese Bibliothek ist Teil der C++-Standardbibliothek, welche eine Sammlung an generischen Containern, Algorithmen und vielen häufig genutzten Funktionen ist.
Um auf die Elemente dieser Bibliothek zuzugreifen, muss man ihren \lstinline{namespace} (in diesem Fall \lstinline{std}) voranstellen, gefolgt von zwei Doppelpunkten und dem gewünschten Element (in diesem Fall \lstinline{cout} und \lstinline{endl}).
Um Überschneidungen mit eigenen Definitionen zu vermeiden, ist es üblich Bibliotheken in einem \lstinline{namespace} zu kapseln, welcher analog zu \lstinline{package} in Java funktioniert, jedoch nicht an Ordnerstrukturen gebunden ist.

In der dritten Zeile wird der String \lstinline{"Hello World"} in \lstinline{std::cout} eingefügt, gefolgt von \lstinline{std::endl}, das einen Zeilenumbruch erzeugt und die Ausgabepuffer leert. Für weitere Informationen zur Kommandozeilenausgabe siehe \url{http://www.cplusplus.com/doc/tutorial/basic_io/} und \url{http://www.cplusplus.com/reference/iomanip/}.

\hints{
	\item Einzeilige Kommentare können durch \lstinline{//}, mehrzeilige durch \lstinline{/* ... */} eingeschlossen werden.
	\item Anders als in Java können Funktionen auch außerhalb von Klassen definiert und verwendet werden.
	\item Die \lstinline{return} Anweisung darf in der \lstinline{main} Funktion weggelassen werden.
}

\subsection*{Häufige (Compiler-)Fehlermeldungen}

Im Folgenden sind einige Fehlermeldungen von \texttt{clang} zusammen mit möglichen Lösungsstrategien aufgelistet.
Die generelle Faustregel lautet: 
\textbf{Kompilierfehler sollten immer von oben nach unten abgearbeitet werden, so wie sie in der Konsole erscheinen.}
Der Grund hierfür ist, dass es durch einen Fehler zu weiteren Folgefehlern kommen kann.

% Indents all of the following paragraphs by 1cm

\begin{verbatim}
  main.exe: not found
\end{verbatim}

Dieser Fehler wird von CodeLite geworfen, wenn es nach dem Kompilieren das lauffähige Programm nicht findet.
Das kann zwei Gründe haben:
\begin{itemize}
\item Der Kompiliervorgang ist gescheitert. Prüfe die \texttt{Console} auf Fehler.
\item Der Kompiliervorgang wurde noch nicht ausgeführt. Kompiliere das Programm mit einem Klick auf das Build-Symbol.
\end{itemize}

\begin{verbatim}
  error: expected ';' before ...
\end{verbatim}

Dies bedeutet, dass in der Zeile davor ein \textbf{;} vergessen wurde.
Allgemein beziehen sich Fehlermeldungen \textbf{expected ... before ...} häufig auf die Zeile \textbf{vor} dem markierten Statement.
Beachte, dass \emph{die Zeile davor} auch die letzte Zeile einer eingebundenen Header-Datei sein kann. Beispiel:

\lstinputlisting{01_basics/problems/listings/helloworld_error_main.cpp}

Falls im Header \filename{main.h} in der letzten Zeile ein Semikolon fehlt, wird der Compiler die Fehlermeldung trotzdem auf die Zeile \glqq{}\lstinline{int main() {}\grqq{} beziehen!!

\begin{verbatim}
  error: invalid conversion from <A> to <B>.
\end{verbatim}

Dies bedeutet, dass der Compiler an der entsprechenden Stelle einen Ausdruck vom Typ \emph{B} erwartet, im Code jedoch ein Ausdruck vom Typ \emph{A} angegeben wurde. Insbesondere bei verschachtelten Typen sowie (später vorgestellten) Zeigern und Templates kann die Fehlermeldung sehr lang werden. In so einem Fall lohnt es sich, den Ausdruck in mehrere Teilausdrücke aufzubrechen und die Teilergebnisse durch temporäre Variablen weiterzureichen.

\begin{verbatim}
  undefined reference to ...
\end{verbatim}

Dies bedeutet, dass das Programm zwar korrekt kompiliert wurde, der Linker aber die Definition des entsprechenden Bezeichners nicht finden kann.
Das kann passieren, wenn man dem Compiler durch einen Prototypen mitteilt, dass eine bestimmte Funktion existiert (\textbf{deklariert}), diese aber nirgendwo tatsächlich \textbf{definiert}.
Überprüfe in diesem Fall, ob der Bezeichner tatsächlich definiert wurde und ob die Signatur der Definition mit dem Prototypen übereinstimmt.

\newpage
\section{C++ Grundlagen, Funktionen und Strukturierung}
Für diese Aufgabe kannst du entweder das vorherige Programm weiter entwickeln oder genauso wie vorher ein neues Projekt anlegen.

\subsection*{Primitive Datentypen} 
Die primitiven Datentypen in C++ sind ähnlich denen in Java.
Allerdings sind alle Ganzzahl-Typen in C++ sowohl mit als auch ohne Vorzeichen verfügbar.
Standardmäßig sind Zahlen vorzeichenbehaftet.
Mittels \lstinline{unsigned} kann man vorzeichenlose Variablen deklarieren.
Durch das freie Vorzeichenbit kann ein größerer positiver Wertebereich dargestellt werden.

\begin{lstlisting}
int i;						// signed int, -2147483648 to +2147483647 on a 32-bit machine
unsigned int ui;			// unsigned int, 0 to 4294967295 on a 32-bit machine
// unsigned double d;	// not possible
\end{lstlisting}

Eine andere Besonderheit von C++ ist, dass Ganzzahlwerte implizit in Boolesche Werte (Typ: \lstinline{bool}) umgewandelt werden.
Alles ungleich 0 wird als \lstinline{true} gewertet, 0 als \lstinline{false}.
Somit können Ganzzahlen direkt in Bedingungen ausgewertet werden.

\subsection{Größe von Datentypen}
Die Größe von den verschiedenen Datentypen ist essentiell zu wissen, wenn man mit ihnen arbeiten möchte.
Deshalb sollst du dir in dieser Aufgabe die Größe der folgenden Datentypen in Bits, wie auch deren minimalen und maximalen Wert ausgeben lassen.

\begin{lstlisting}
    int
    unsigned int
    double
    unsigned short
    bool
\end{lstlisting}

\hints{
    \item Zum Überprüfen der Größe von Datentypen kann man den \lstinline{sizeof()} Operator\footnote{\url{http://en.cppreference.com/w/c/language/sizeof}} verwenden.
    \item Die C++ Klasse \lstinline{std::numeric\_limits}\footnote{\url{http://en.cppreference.com/w/cpp/types/numeric_limits}} bietet Funktionen sich minimale und maximale Werte von Datentypen ausgeben zu lassen. 
Einbinden lässt sich diese über den Header \lstinline{limits}.
}

\subsection{Sternmuster mit Funktionen malen}

Schreibe eine Funktion \lstinline{printStars(int n)}, die \lstinline{n}-mal ein * auf der Konsole ausgibt und mit einem Zeilenumbruch abschließt.
Ein Aufruf von \lstinline{printStars(5)} sollte folgende Ausgabe generieren:

\begin{lstlisting}
*****
\end{lstlisting}

Platziere die Funktion \textbf{vor} der \lstinline{main}, da sie sonst von dort aus nicht aufgerufen werden kann.
Benutze die erstellte Funktion \lstinline{printStars(int n)}, um eine weitere Funktion zu schreiben, die eine Figur wie unten dargestellt ausgibt.
Verwende hierzu Schleifen.

\begin{lstlisting}
*****
****
***
**
*
**
***
****
*****
\end{lstlisting}

\hints{
    \item Was die Benennung von Funktionen, Variablen und Klassen angeht, bist du frei. Für Klassen ist \enquote{CamelCase} wie in Java üblich. Bei Funktionen und Variablen wird zumeist entweder auch Camel Case oder Kleinschreibung mit Unterstrichen verwendet.
    \item Um Strings auszugeben, stellt dir C++ \lstinline{std::cout} zur Verfügung, welches den String zu dem Standart Output Stream weitergibt. Diesen Output Stream kann man mit dem Manipulator \lstinline{std::endl} zu einem Zeilenumbruch zwingen.
}

\subsection{Auslagern der Datei}
Erstelle eine neue Header-Datei \filename{functions.hpp} und eine neue
Sourcedatei \filename{functions.cpp}.
Klicke hierzu mit der rechten Maustate auf den Ordner \textbf{src} und wähle \textbf{New class}, gib einen Klassennamen ein und bestätige den Dialog mit \textbf{OK}. Alle anderen Felder werden automatisch ausgefüllt oder können leer bleiben.
Beachte hierbei, dass CodeLite automatisch Include-Guards in der Headerdatei erzeugt, die das mehrfache Einbinden desselben Headers verhindern:

\begin{lstlisting}
#ifndef FUNCTIONS_HPP_
#define FUNCTIONS_HPP_
// your header ...
#endif /* FUNCTIONS_HPP_ */
\end{lstlisting}

Binde danach \filename{functions.hpp} in beide Sourcedateien ein indem du

\begin{lstlisting}
#include "functions.hpp"
\end{lstlisting}

verwendest.
\textbf{Verschiebe} deine beiden Funktionen nach \filename{functions.cpp}.

Schreibe nun in \filename{functions.hpp} \textbf{Funktionsprototypen} für die beiden Funktionen aus der vorherigen Aufgabe.
Funktionsprototypen dienen dazu, dem Compiler mitzuteilen, dass eine Funktion mit bestimmtem Namen, Parametern und Rückgabewert existiert.
Ein Prototyp ist im wesentlichen eine mit \textbf{;} abgeschlossene Signatur der Funktion ohne Funktionsrumpf.
Der Prototyp von \lstinline{printStars(int n)} lautet 
\begin{lstlisting}
void printStars(int n);
\end{lstlisting}

Fertig -- die Ausgabe des Programms sollte sich nicht verändert haben.

\hints{
	\item Sourcedateien tragen in der Regel die Endung \filename{.cpp}, Headerdateien \filename{.h} oder \filename{.hpp}.
	\item Denke daran, auch in \filename{functions.cpp} den Header \lstinline{iostream} einzubinden, falls du dort Ein- und Ausgaben verwenden willst (\lstinline{#include<iostream>}).
	\item Beachte, dass es zwei verschiedene Möglichkeiten gibt, eine Header-Datei einzubinden - per \lstinline{\#include <Bibliotheksname>} sowie per \lstinline{\#include "Dateiname"}.
    Bei der ersten Variante sucht der Compiler nur in den Include-Verzeichnissen der Compiler-Toolchain, während bei der zweiten Variante auch die Projektordner durchsucht werden. Somit eignet sich die erste Schreibweise für System-Header und die zweite für eigene, projektspezifische Header.
	\item Anstelle der Include-Guards kannst du auch die Präprozessordirektive \lstinline{\#pragma once} verwenden. Dies wird von den meisten Compilern unterstützt.
}

\subsection{Dokumentation} \label{basics:doc}
Für die Lesbarkeit eines Programms ist eine ausführliche Dokumentation des Programmcodes essentiell.
Damit ihr schonmal einen Einblick darin bekommt, werden wir hier in dem Praktikum mit kommentieren und das Tool \emph{Doxygen}\footnote{Doxygen-Link als Referenz: \url{http://www.doxygen.nl/}} zur Erstellung von Dokumentationen verwenden.

Damit Doxygen deine Kommentare erkennt, müssen sie ein spezielles Format einhalten.
\begin{itemize}
    \item Die Kommentare vor den jeweiligen zu kommentierenden Elementen (z.B. Funktionen) stehen.
    \item{Mehrzeilige Kommentare müssen den folgenden Stil einhalten (beachte hierbei das zusätzliche \lstinline{*} in der ersten Zeile)
        \begin{lstlisting}
/**
 * Content
 */
        \end{lstlisting}}
\end{itemize}

Außerdem müssen bestimmte Kommandos in den Kommentaren verwendet werden, die Doxygen bei der Dokumentationsgenerierung verwenden kann.
Diese Kommandos sind die folgenden

\begin{itemize}
    \item \lstinline{@brief kurzeBeschreibung} einzeilige Beschreibung des Elements.
    \item \lstinline{@author AuthorenNamen} für den Namen des Autoren.
    \item \lstinline{@name Funktionsnamen} für den Namen der Funktion.
    \item \lstinline{@param Parametername Beschreibung} um Parameterübergaben bei Funktionen zu erklären.
    \item \lstinline{@return Beschreibung} kurze Beschreibung der Rückgabe.
    \item \lstinline{@file Dateiname} damit Doxygen den kompletten File parst.
\end{itemize}

Da das Dokumentieren der Datei nicht vor der Datei passieren kann (wo sollte das sein?), geschieht es deshalb direkt nach den Präprozessor Direktiven.

Eure Aufgabe ist es nun, den von euch erstellten Code sorgfältig zu dokumentieren.
Die Dokumentation geschieht dabei in der \filename{.h} oder \filename{.hpp} Datei.

Hier ein kleines Beispiel dazu, damit ihr eine Vorstellung davon bekommt, wie das ganze am Ende auszusehen hat.

\begin{lstlisting}
#ifndef TESTING_HPP_
#define TESTING_HPP_

/**
 * @file testing.hpp
 * @author Your Name
 * @brief Just a showcase hpp file to demonstrate doxygen comments
 */

/**
 * @name first_func(int a);
 * @author Your Name
 * @brief A showcase function.
 * @param a Used for important stuff.
 * @return void
 */
void first_func(int a);

/**
 * @name second_(int a, char b, double d);
 * @author Your Name
 * @brief A showcase function.
 * @param a Used for important stuff.
 * @param b Used for important stuff.
 * @param d Used for important stuff.
 * @return void
 */
void second_func(int a, char b, double d);

#endif /* TESTING_HPP_ */
\end{lstlisting}

Erstellen könnt ihr die Dokumentation am Ende über die Kommandozeile.
Dafür öffnet ihr euer Terminal mit \lstinline{STRG + ALT + t} und wechselt in das Verzeichnes in dem euer Projekt liegt (üblicherweise \lstinline{cd CPPP/Workspace/NameEuresProjektes}).
Dort gebt ihr \lstinline{doxygen -g} ein, welches euch eine vorgefertigte Konfigurationsdatei von \texttt{doxygen} erstellt, gefolgt von \lstinline{doxygen}, was letztendlich die Dokumentation erstellt.
Diese findet ihr in dem Ordner \filename{html} unter der Datei \filename{index.html}.
Ihr erreicht die Datei ganz einfach, wenn ihr in der Terminal das Kommando \lstinline{pcmanfm} eingebt, was euren Filemanager öffnet.
Auf der Webseite, die sich dann öffnet, findet ihr unter Files die von euch dokumentierte \filename{functions.{h,hpp}}.

\subsection{Eingabe}
Erweitere das Programm um eine Eingabeaufforderung zur Bestimmung der Breite der auszugebenden Figur.
Die Breite soll dabei eine im Programmcode vorgegebene Grenze nicht überschreiten dürfen.
Gib gegebenenfalls eine Fehlermeldung aus.
Verwende zum Einlesen \lstinline{std::cin} und \lstinline{operator>>} wie in folgendem Beispiel.

\begin{lstlisting}
int x;
std::cin >> x; // Type, e.g., 174 and press ENTER.
// Now, x contains the entered number.
std::cout << x << std::endl.
\end{lstlisting}

Erstelle auch für diesen Aufgabenteil eine eigene Funktion und lagere diese nach \filename{functions.cpp} aus.

\subsection{Fortlaufendes Alphabet ausgeben}
Statt eines einzelnen Zeichens soll nun das fortlaufende Alphabet ausgegeben werden.
Sobald das Ende des Alphabets erreicht wurde, beginnt die Ausgabe erneut bei \emph{a}.
Beispiel:

\begin{lstlisting}
abc
de
f
gh
ijk
\end{lstlisting}

Implementiere dazu eine Funktion \lstinline{char nextChar()}.
Diese soll bei jedem Aufruf das nächste auszugebende Zeichen von Typ \lstinline{char} zurückgeben, beginnend bei \emph{a}.
Dazu muss sich \lstinline{nextChar()} intern das aktuelle Zeichen merken.
Dies kann durch die Verwendung von statischen Variablen erreicht werden. Diese behalten ihren alten Wert beim Wiedereintritt in die Funktion.
Ein statische Variable \lstinline{c} wird mittels

\begin{lstlisting}
static char c = 'a';
\end{lstlisting}

deklariert.
In diesem Fall wird die Variable \lstinline{c} \textbf{einmalig zu Beginn des Programms} mit \lstinline{'a'} initialisiert und kann später beliebig verändert werden.

\hints{
	\item Der Datentyp \lstinline{char} kann wie eine Zahl verwendet werden, d.h. man kann z.B. die Modulooperation \lstinline{\%} verwenden.
}

\subsection{Namensräume}
Bibliotheken werden in einen eigenen Namensraum gekapselt, damit ihre Funktionen nicht mit gleichnamigen Funktionen in anderen Bibliotheken kollidieren.
Erweitere dazu das Programm, indem du im Header die Funktionsprototypen wie
folgt in einen \lstinline{namespace} setzt.

\begin{lstlisting}
namespace fun {
	// function prototypes ...
};	// semicolon!
\end{lstlisting}

Denke daran, dass du die Namen der Funktionen in der Sourcedatei noch anpassen musst, indem du vor jede Funktion den gewählten \lstinline{namespace}-Namen gefolgt von zwei Doppelpunkten setzt.
Genauso muss der Namensraum vor jedem Aufruf der Funktion gesetzt werden.

\begin{lstlisting}
void fun::print_star(int n) {
	// ...
}
\end{lstlisting}

Vergisst man, den Namensraum in der Sourcedatei anzugeben, findet der Linker keine Implementation zu der im Header definierten Funktion.
Weiterhin stünde diese Funktion nicht mehr im Bezug zum Header und könnte nur noch lokal verwendet werden (\lstinline{print_star(int n)} und \lstinline{fun::print_star(int n)} sind unterschiedliche Funktionen!). \\ \smallskip

Falls man seine Funktionen noch weiter unterteilen möchte, kann man Namensräume auch schachteln.
Hierzu definiert man wie oben ein weiteren Namensraum mit \lstinline{namespace} in einer anderen Namensraum Instanz.

\begin{lstlisting}
namespace fun {
    namespace ny{
	// function prototypes ...
    };	// semicolon!
};	// semicolon!
\end{lstlisting}

In der Sourcedatei folgt dann nach dem initialen Namensraum, hier \lstinline{fun} gefolgt von dem Doppelpunktpaar, der geschachtelte Namensraum \lstinline{ny} wieder gefolgt von zwei Doppelpunkten.
Erst dann können die Funktionen in dem Namensraum \lstinline{ny} verwendet werden.

\begin{lstlisting}
void fun::ny::print_star(int n) {
	// ...
}
\end{lstlisting}

In diesem Projekt wird dies nicht notwendig sein, da die Anzahl der definierten Funktionen überschaubar ist, aber trotzdem empfehlen wir dir es auszuprobieren.

\hints{
	\item Du kannst \lstinline{using namespace fun;} verwenden, um diesen Namensraum zu importieren (vergleichbar mit \lstinline{static import} in Java).
    Genauso wie in Java kann es hierdurch leichter zu Namenskollisionen kommen und sollte daher eher nicht verwendet werden.
    \item Bei dem geschachtelten Namensraum ist entsprechend \lstinline{using namespace fun::ny;} zu verwenden um den Namensraum zu importieren.
}

\newpage
\section{\ExercisePrefixBasics Klassen}
Ziel dieser Aufgabe ist es, die vorherige Aufgabe objektorientiert zu lösen. Schreibe hierfür manuell eine Klasse, die das aktuelle Zeichen als Attribut enthält und durch Methoden ausgelesen und inkrementiert werden kann.

\hints{
	\item Verwende in dieser Aufgabe noch \textbf{nicht} den Klassengenerator (\textbf{Rechtsklick auf den Ordner src/ $\rightarrow$ New class...}) von CodeLite!
}

\subsection{Definition}
Eine Klasse wird üblicherweise analog zu der vorherigen Aufgabe in Deklaration (Headerdatei) und Implementation (Sourcedatei) aufgeteilt.
Die Struktur der Klasse mit allen Attributen und Funktionsprototypen wird im Header beschrieben, während die Sourcedatei nur die Implementation der Funktionen und Initialisierungen statischer Variablen enthält.
Standardmäßig sind alle Elemente einer Klasse privat.
Im Gegensatz zu Java werden in C++ die Access-Modifier \lstinline{public}/\lstinline{private}/\lstinline{protected} nicht bei jedem Element einzeln, sondern blockweise angegeben.

  \lstinputlisting{01_basics/problems/listings/classes_intro.hpp}

Erzeuge einen Header \filename{CharGenerator.hpp} und erstelle den Klassenrumpf der Klasse \lstinline{CharGenerator}.
Füge der Klasse das \lstinline{private} Attribut \lstinline{char nextChar} hinzu, in dem das als nächstes auszugebende Zeichen gespeichert wird und einen \lstinline{public} Konstruktorprototypen \lstinline{CharGenerator()}, der \lstinline{nextChar} auf \lstinline{'a'} initialisieren soll.
Füge noch einen \lstinline{public} Funktionsprototypen \lstinline{char generateNextChar()} hinzu, welcher das nächste auszugebende Zeichen zurückgeben soll.


\hints{
	\item Ein Konstruktor wird als eine Funktion ohne Rückgabetyp deklariert, die den gleichen Namen wie die Klasse hat, und beliebige Parameter beinhalten kann.
}

\subsection{Dokumentation von Klassen}
Auch in dieser Aufgabe geht es wieder darum, deinen Programmcode zu dokumentieren.
Das funktioniert wieder sehr ähnlich wie in Aufgabe \ref{basics:doc}, nur tauschst du den Tag \lstinline{@file} gegen \lstinline{@class} aus und platzierst die Dokumentation vor der Definition der Klasse.
Dies kannst du fortlaufend in der Aufgabe erfüllen, es muss nicht direkt jetzt geschehen.
Am Ende erstellst du dir wieder eine Dokumentation der Klassen und kannst so entdecken, wie \texttt{doxygen} deine Kommentare in eine fertige Dokumentation umsetzt.

\subsection{Implementation}
Wie bei der Verwendung von \lstinline{namespace} muss der Scope der Klasse (der Klassenname) in der Sourcedatei vor jeder Elementbezeichnung (Konstruktor, Funktion, \dots) durch zwei Doppelpunkte (den Scope-Resolution-Operator) getrennt angegeben werden.

  \lstinputlisting{01_basics/problems/listings/classes_impl_func.cpp}

Um Attribute zu initialisieren, wird üblicherweise eine sogenannte Initialisierungsliste im Konstruktor verwendet, da diese vor dem Eintritt in den Konstruktorrumpf aufgerufen wird.
Die Initialisierungsliste wird durch einen Doppelpunkt zwischen der schließenden Klammer der Parameterliste und der öffnenden geschweiften Klammer des Rumpfes eingeleitet, und bildet eine mit Komma separierte Liste von Attributnamen und ihren Initialisierungsargumenten in Klammern.

  \lstinputlisting{01_basics/problems/listings/classes_impl_list.cpp}

Erzeuge eine Sourcedatei \filename{CharGenerator.cpp} für die Implementation der Klasse und binde die \filename{CharGenerator.hpp} ein.
Implementiere den Konstruktor, indem du \lstinline{nextChar} mit \lstinline{'a'} in der Initialisierungsliste initialisierst.
Implementiere zudem \lstinline{generateNextChar()}, indem du \lstinline{nextChar} zurückgibst.

\hints{
	\item Die Reihenfolge der Initialisierungsliste sollte der Deklarationsreihenfolge entsprechen.
	\item Konstanten \textbf{müssen} in der Initialisierungsliste zugewiesen werden, damit diese zur Laufzeit bekannt sind.
}

\subsection{Instantiierung}
Erzeuge wie aus den vorherigen Aufgaben bekannt eine \filename{main.cpp} mit einer \lstinline{main()}-Funktion in der du ein \lstinline{CharGenerator}-Objekt erzeugst und \lstinline{generateNextChar()} mehrfach aufrufst und ausgibst.

  \lstinputlisting{01_basics/problems/listings/classes_obj.cpp}

Überprüfe das Ergebnis über die Konsole oder den Debugger.

\hints{
	\item Um ein Objekt zu erzeugen, muss in C++ kein \lstinline{new} verwendet werden (siehe dazu nächste Vorlesung).
}

\subsection{Default-Parameter}
Damit man nicht immer das Startzeichen angeben muss, kann man einen Default-Wert für einen Parameter angeben. Beim Aufruf kann dieser Parameter dann weggelassen werden kann.
Hierzu wird dem Parameter im Prototypen (im Header) ein Wert zugewiesen, ohne die Implementation zu ändern.

  \lstinputlisting{01_basics/problems/listings/classes_default_value.cpp}

Erweitere den Konstruktor um einen Parameter \lstinline{char initialChar}, welcher defaultmäßig \emph{a} ist und ändere die Initialisierung von \lstinline{nextChar}, damit dieser mit dem übergebenen Parameter gestartet wird.

Teste deine Implementation sowohl mit als auch ohne Angabe des Startzeichens.
Um ein Startzeichen anzugeben, lege das Objekt wie folgt an:

  \lstinputlisting{01_basics/problems/listings/classes_obj_par.cpp}

\hints{
	\item Bei der Definition eines Default-Parameters müssen für alle nachfolgenden Parameter ebenfalls mit Default-Werten angegeben werden, um Mehrdeutigkeiten beim Aufruf zu vermeiden.
}

\subsection{PatternPrinter}
Implementiere folgende Klasse.

  \lstinputlisting{01_basics/problems/listings/classes_pattern_printer.hpp}

Teste deine Implementation, indem du ein \lstinline{PatternPrinter}-Objekt anlegst und \lstinline{printPattern()} darauf aufrufst.

\hints{
	\item Ohne eine Initialisierungsliste wird \lstinline{charGenerator} mit dem Default-Parameter initialisiert.
	Um ein eigenes Startzeichen anzugeben, muss eine Initialisierungsliste erstellt und \lstinline{charGenerator} mit dem entsprechenden Argument initialisiert werden.
}

\newpage
\section{\ExercisePrefixBasics Operatorenüberladung}\label{sec:overloading}
\label{sec:overloading}
\cpppSolutionName{operator_overloading}{operator\_ overloading}
In C++ besteht die Möglichkeit, Operatoren wie \textbf{+} (\lstinline{operator+}), \textbf{*} (\lstinline{operator*}), \dots zu überladen.
Man kann selber spezifizieren, was beim Verknüpfen von Objekten mit einem Operator geschehen soll, um zum Beispiel den Quellcode übersichtlicher zu gestalten.
Du hast bereits das Objekt \lstinline{std::cout} der Klasse \lstinline{std::ostream} kennengelernt, welche den $<<$-Operator überlädt, um Ausgaben von \lstinline{std::string}, \lstinline{int}, \dots komfortabel zu tätigen.
In dieser Aufgabe sollst du eine eigene Vektor-Klasse schreiben und einige Operatoren überladen.

\hints{
    
    \item Ausführliche Hinweise zum Überladen von Operatoren findest du hier: \url{http://en.cppreference.com/w/cpp/language/operators}.
}

\subsection{Konstruktor und Destruktor}
Implementiere die folgende Klasse.
Füge jedem Konstruktor und Destruktor eine Ausgabe auf der Konsole hinzu, um beim Programmlauf den Lebenszyklus der Objekte nachvollziehen zu können.

  \cpppInputListing{01_basics/problems/listings/overloading_vector.hpp}

Der Copy-Konstruktor wird aufgerufen, wenn das Objekt kopiert werden soll, z.B. für eine Call-by-Value Parameterübergabe.
Jeder Copy-Konstruktor benötigt eine Referenz auf ein Objekt vom gleichen Typ wie die Klasse selbst als Parameter. 
Sinnvollerweise wird noch \lstinline{const} vor oder nach der Typbezeichnung eingefügt (aber vor \lstinline{&}), da typischerweise das Ursprungsobjekt nicht verändert wird.

Der Destruktor wird aufgerufen, sobald die Lebenszeit eines Objekts endet. Er wird verwendet, um Ressourcen, die das Objekt besitzt, freizugeben.
Die Syntax des Prototypen lautet

  \cpppInputListing{01_basics/problems/listings/overloading_destructor.hpp}

und die Implementation entsprechend

  \cpppInputListing{01_basics/problems/listings/overloading_destructor_impl.cpp}

\hints{
	\item Es darf eine beliebige Anzahl an Konstruktoren mit verschiedenen Parametersätzen existieren.
	\item Der Compiler wird automatisch einen \lstinline{public} Destruktor und \lstinline{public} Copy-Konstruktor erzeugen, falls sie nicht \emph{deklariert} wurden.
    Ebenso wird ein \lstinline{public} Defaultkonstruktor (keine Argumente) automatisch vom Compiler generiert, falls überhaupt keine Konstruktoren deklariert wurden.
    
    Falls du sie jedoch \emph{deklarierst}, musst du auch eine Implementierung angeben.
	\item Würden beim Copy-Konstruktor \lstinline{other} by-Value übergeben werden, müsste eine Kopie von \lstinline{other} angelegt werden.
	Dazu würde der Copy-Konstruktor aufgerufen, was zu einer unendlichen Rekursion führt, bis der Stack seine maximale Größe überschreitet und das Programm abstürzt.
}

\subsection{Vektoraddition, Vektorsubtraktion und Skalarprodukt}
Erweitere die Klasse um folgende \lstinline{public} Funktionen, um Vektoren durch \lstinline{v1 + v2}, \lstinline{v1 - v2} und \lstinline{v1 * v2} addieren/subtrahieren und das Skalarprodukt bilden zu können, indem die Operatoren $+$, $-$ und $*$ überladen werden.

  \cpppInputListing{01_basics/problems/listings/overloading_operators.hpp}

Innerhalb der Methode kannst du durch \lstinline{a}, \lstinline{b} und \lstinline{c} auf eigene Attribute und über  \lstinline{rhs.a}, \lstinline{rhs.b} und \lstinline{rhs.c} auf Attribute der rechten Seite zugreifen.
Denke daran, bei der Implementation der Klassen den Scope der Klasse in der Sourcedatei vor jeder Elementbezeichnung durch zwei Doppelpunkte getrennt (den bereits bekannten Scope-Resolution-Operator) anzugeben.

  \cpppInputListing{01_basics/problems/listings/overloading_operators_impl.cpp}

\hints{
	\item Der Parameter \lstinline{rhs} steht für die rechte Seite ( \enquote{right-hand-side}) des jeweiligen Operators.
	Dadurch, dass der Operator als Member der Klasse deklariert wurde, nimmt die aktuelle Instanz hierbei automatisch die linke Seite der Operation (\enquote{left-hand side}) an.
	\item Der Rückgabetyp eines Skalarprodukts (dot product) ist kein \lstinline{Vector3} sondern ein Skalar (\lstinline{double})!
}

\subsection{Ausgabe}
Überlade den \lstinline{operator<<} zur Ausgabe eines Vektors mit der gewohnten \lstinline{std::cout << ...} Syntax, indem du den folgenden Funktionsprototypen \textbf{außerhalb} der Klassendefinition setzt

  \cpppInputNoPageBreakListing{01_basics/problems/listings/overloading_stdout.hpp}

und innerhalb der Sourcedatei wie folgt implementierst.

  \cpppInputNoPageBreakListing{01_basics/problems/listings/overloading_stdout_impl.cpp}

Da der \lstinline{operator<<} außerhalb der Klasse \lstinline{Vector3} liegt, hat dieser keinen Zugriff auf die privaten Member der Klasse.
Du hast zwei Möglichkeiten, Zugriff auf diese zu erlangen: per Getter und per \lstinline{friend}-Deklaration.

\paragraph{Getter}
Definiere die folgenden Gettermethoden, die die Werte für die \lstinline{private} Attribute \lstinline{a}, \lstinline{b} und \lstinline{c} zurückgeben:
  \cpppInputNoPageBreakListing{01_basics/problems/listings/overloading_getter.hpp}

\paragraph{friend}
Füge die folgende Zeile am Ende der Klasse \texttt{Vector} hinzu:

  \cpppInputNoPageBreakListing{01_basics/problems/listings/overloading_friend.hpp}
Von nun an kann die entsprechende Funktion auf alle privaten Member der Klasse \texttt{Vector} zugreifen, was insbesondere praktisch ist, falls die Klasse verändert werden soll.

\hints{
	\item Denke daran, den Header \lstinline{iostream} einzubinden.
	\item Diesmal musste die Überladung \textbf{außerhalb} der \lstinline{Vektor3}-Klasse definiert werden, weil das \lstinline{Vektor3}-Objekt auf der rechten Seite der Operation steht.
	Als linke Seite wird hierbei ein \lstinline{std::ostream}-Objekt (wie z.B. \lstinline{std::cout}) erwartet, um Ausgabeketten \lstinline{std::cout << ... << ...} zu ermöglichen.
	Hierzu muss das Ausgabeobjekt auch zurückgegeben werden. Damit das \lstinline{std::ostream}-Objekt aber nicht jedes Mal kopiert wird, wird es als Referenz \lstinline{&} durchgereicht.
	\item Anstatt Getter und Setter für \lstinline{private} Attribute zu schreiben, kann man auch einer Klasse oder Funktion vollen Zugriff mit Hilfe des Schlüsselworts \lstinline{friend} erlauben. In der nächsten Übung wird hierauf noch einmal eingegangen.
}


\subsection{Testen}
Teste deine bisher definierten Methoden und Funktionen.
Probiere auch Kombinationen von verschiedenen Operatoren aus und beobachte das Ergebnis.
Schreibe auch eine einfache Funktion, die Vektoren als Parameter nimmt.
Wie du siehst, werden sehr viele \lstinline{Vector3}-Objekte erstellt, kopiert und gelöscht.
Dies liegt daran, dass die Objekte immer per Call-by-Value übergeben und dabei kopiert werden.
Wie dies vermieden werden kann, siehst du im Themenbereich \enquote{Speicherverwaltung}.


\cclicense

\end{document}
