\section*{Einführung}
Von der Toolseite aus unterscheidet sich die Entwicklung für den Mikrocontroller kaum von den bisherigen Übungen.
Der Aufruf der Toolchain (Compiler, Linker und Flashprogramm) für die Embedded-Entwicklung ist in einem Makefile integriert, das in CodeLite durch den Buildprozess automatisch aufgerufen wird.

Verwende für jede Aufgabe das zugehörige Vorlageprojekt aus dem Übungsrepository (\url{https://github.com/Echtzeitsysteme/tud-cpp-exercises}), da diese sowohl das Makefile, als auch notwendige Bibliotheken im Ordner \filename{uc\_includes} zur Verwendung des Boards enthalten.

\hints{
	\item Sollte der Flashvorgang fehlschlagen, kann es helfen den Schiebeschalter auf dem Board auf \textbf{PROG} zu schalten.
	Nach dem Flashvorgang setzt man dann den Schiebeschalter auf \textbf{RUN} und startet das Programm mit Hilfe der blauen Resettaste.
	\item Pool-PCs: Das USB-Kabel muss am Hub in der Buchse stecken, die mit \emph{Data} beschriftet ist.
	\item Variablen dürfen in dieser Übung im Gegensatz zu C++ nur am Anfang eines Anweisungsblocks deklariert werden.
	Erst ab dem C99 Standard (welche hier nicht verwendet wird) dürfen Deklarationen an beliebigen Stellen erfolgen.
}
