\section{Buttons}
Implementiere einen Counter der bei 00 startet und bis 99 zählen kann.
Bei Druck auf die rechte Taste soll der Wert erhöht, bei Druck auf die linke Taste verringert werden.
Nutze die Siebensegmentanzeige aus der vorherigen Aufgabe zur Anzeige des aktuellen Wertes.

Wird 99 angezeigt und die rechte Taste gedrückt, soll der Counter auf 00 gesetzt werden.
Umgekehrt gilt: Falls der Counter 00 anzeigt und die linke Taste gedrückt wird, soll er auf 99 umspringen.

Der linke Taster ist an Port 07 Pin 0, der rechte an Port 07 Pin 1 angeschlossen.
Bei gedrücktem Taster liegt ein Low-Pegel am Eingang, sonst ein High-Pegel.
Du kannst den Zustand eines Pins wie folgt abfragen:
\lstinputlisting{problems/listings/buttons.c}

\hints{
	\item Ein Button ist üblicherweise für mehrere tausend CPU-Zyklen gedrückt.
	\item Ein Tastendruck ist durch den Übergang von \emph{high} auf \emph{low} definiert.
	Da in jedem Zyklus nur der aktuelle Wert abgefragt werden kann, musst du den aktuellen Wert mit einem gespeicherten Wert aus dem vorigen Durchlauf vergleichen.
	\item Die Moduloberechnung von negativen Zahlen kann problematisch sein.
	Hier hilft es, wenn du vor der Berechnung den Divisor hinzuaddierst.
}
