\section{\ExercisePrefixObjectOrientation Mehrfachvererbung}
\cpppSolutionName{multiple_inheritance}{multiple\_inheritance}
Verwende den Code der \ref{sec:inheritance} als Basis.

\subsection{Klasse \lstinline{Employee}}
Schreibe die Klasse \lstinline{Employee}, die einen Mitarbeiter darstellt.
\lstinline{Employee} soll von \lstinline{Person} erben und den Namen seines Vorgesetzten als Attribut beinhalten.
Erweitere auch entsprechend die Methode \lstinline{getInfo()}.

\subsection{Klasse \lstinline{StudentAssistant}}
Schreibe nun eine Klasse \lstinline{StudentAssistant}, die eine wissenschaftliche Hilfskraft modelliert.
Eine wissenschaftliche Hilfskraft ist ein Student und gleichzeitig auch ein Mitarbeiter.
Dementsprechend soll \lstinline{StudentAssistant} sowohl von \lstinline{Student} als auch von \lstinline{Employee} erben. 
Das heißt es werden je ein Student- und ein Employee-Objekt im Konstruktor initialisiert.
Weitere Attribute sind nicht nötig.
Überschreibe \lstinline{getInfo()}, um alle Daten auszugeben.
Ändere dazu die Sichtbarkeit der Attribute sowohl von \lstinline{Student} als auch von \lstinline{Employee} von \lstinline{private} auf \lstinline{protected}.

Du wirst feststellen, dass sich die Klasse nicht kompilieren lässt, falls du das Attribut \lstinline{name} direkt verwendest, da in einer \lstinline{StudentAssistant}-Instanz zwei Instanzen von \lstinline{Person} vorhanden sind - je eine von jeder Elternklasse. Deshalb musst du mittels dem Scope-Operator $::$ angeben, welche Klasse du genau meinst.

\cpppInputListing{03_oo/problems/listings/multiInheritance_scope.cpp}

Teste deine Implementation, indem du das Ergebnis von \lstinline{getInfo()} direkt in der \lstinline{main} ausgibst.

\subsection{Virtuelle Vererbung}
Versuche nun, \lstinline{printPersonInfo()} mit einer Instanz von \lstinline{StudentAssistant} aufzurufen. Auch hier wird der Compiler mit einer Fehlermeldung abbrechen, da er nicht weiß, welche der beiden Basisklassen er nehmen soll.
Diesmal ist es in C++ allerdings nicht mehr möglich, die Basisklasse zu spezifizieren, weshalb wir anders vorgehen werden.
Wir sorgen mittels virtueller Vererbung dafür, dass \lstinline{Person} nur ein Mal in \lstinline{StudentAssistant} vorhanden ist.

Lasse dazu \lstinline{Student} und \lstinline{Employee} virtuell von \lstinline{Person} erben.
Noch lässt sich das Programm nicht kompilieren, denn sowohl \lstinline{Student} als auch \lstinline{Employee} versuchen, einen Konstruktor von \lstinline{Person} aufzurufen.
Da \lstinline{Person} aber nur ein einziges mal in \lstinline{StudentAssistant} vorhanden ist, müsste der Konstruktor demnach zwei mal aufgerufen werden -- einmal von \lstinline{Student} und einmal von \lstinline{Employee}.
Dies würde jedoch grob gegen die Sprachprinzipien verstoßen.
Deshalb wird der Konstruktor von \lstinline{Person} weder von \lstinline{Student} noch von \lstinline{Employee} aufgerufen!
Stattdessen müssen wir in der Initialisierungsliste von \lstinline{StudentAssistant} angeben, welcher Konstruktor von \lstinline{Person} aufgerufen werden soll.
Die Konstruktoraufrufe innerhalb von \lstinline{Student} und \lstinline{Employee} laufen stattdessen ins Leere, auch wenn sie syntaktisch vorhanden sind! Füge deshalb ein \lstinline{Person(name)} in die Initialisierungsliste von \lstinline{StudentAssistant} hinzu.

Teste deine Implementation.
Versuche auch Folgendes: Ändere die Namen in den Konstruktoraufrufen von \lstinline{Student} und \lstinline{Employee} in der Initialisierungsliste von \lstinline{StudentAssistant} und beobachte die Ausgabe.
Mache dir dadurch klar, welche Probleme Mehrfachvererbung von implementierten Klassen verursachen kann!

\subsection{Erklärung}

Eine Alternative zur Implementationsvererbung stellt \textbf{Schnittstellenvererbung} dar, wie es in Java üblich ist. Dabei werden Schnittstellen (Klassen mit ausschließlich abstrakten Methoden und ohne Attribute) definiert und nur diese vererbt.
Zusätzlich gibt es Implementationen von diesen Schnittstellen.
Man würde also \lstinline{Person}, \lstinline{Student}, \lstinline{Employee} und \lstinline{StudentAssistant} in jeweils zwei Klassen aufteilen, eine Schnittstelle und eine Implementation.
Die Schnittstellen würden voneinander erben, z.B. \lstinline{StudentBase} von \lstinline{PersonBase}, und entsprechende pur virtuelle/abstrakte Methoden wie \lstinline{virtual std::string StudentBase::GetStudentID() = 0} bereitstellen.
Die Implementation würde ausschließlich von der jeweiligen Schnittstelle erben (\lstinline{Student} von \lstinline{StudentBase}).
Diese Variante erscheint zwar aufwändiger als Implementationsvererbung, vermeidet aber viele der dabei entstehenden Probleme.
Schnittstellenvererbung kann in Java eingesetzt werden, um Mehrfachvererbung zu realisieren.
