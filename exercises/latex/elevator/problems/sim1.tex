\section{Aufzugsimulator -- Tag 1}
In dieser Aufgabe soll ein Grundgerüst für den in der Vorlesung vorgestellten Aufzugsimulator geschaffen werden.
Bei der Bearbeitung dieser Aufgaben geben wir die keinerlei zeitliche Vorgabe.
Du kannst diese Aufgaben direkt an dem Tag lösen, an dem ihr die dafür benötigten Konzepte gelernt habt oder auch nach hinten verschieben und als ausführliche Klausurvorbereitung nutzen.

\subsection{Klasse Person}
Implementiere die Klasse \lstinline{Person}, die eine Person mit einem gewünschten Zielstockwerk darstellt.
Füge allen Konstruktoren und Destruktoren eine Ausgabe auf die Konsole hinzu, um später den Lebenszyklus der Objekte besser nachvollziehen zu können.

\lstinputlisting{problems/listings/sim1_person.hpp}

\subsection{Klasse Elevator}
Implementiere die Klasse \lstinline{Elevator}, die einen Aufzug mit einer beliebigen Anzahl an Personen darstellt.
%
Wenn sich der Aufzug bewegt, solltest du die verbrauchte Energie bei einer Bewegung sinnvoll anpassen.
Addiere beispielsweise den Betrag der Differenz zwischen dem aktuellen und dem Zielstockwerk hinzu.

\lstinputlisting{problems/listings/sim1_elevator.hpp}

Um die Klasse \lstinline{std::vector} aus der Standardbibliothek zu nutzen musst du noch den Systemheader \lstinline{vector} einbinden.
Der Container \lstinline{std::vector} kapselt ein Array und stellt eine ähnliche Funktionalität wie Javas \lstinline{Vector} Klasse bereit.
Der Typ in spitzen Klammern (\lstinline{<Person>} in \lstinline{std::vector<Person>}) ist ein Template-Parameter und besagt, dass in dem Container \lstinline{Person}-Objekte gespeichert werden sollen.

Folgende Funktion der Klasse \lstinline{std::vector} könnten dir von Nutzen sein. Weitere findest du z.B. unter \url{http://www.cplusplus.com/reference/vector/vector/}. Der Typ \lstinline{size_type} ist der größtmögliche vorzeichenloser Integer, die die verwendete Plattform unterstützt.

\lstinputlisting{problems/listings/sim1_sizetype.cpp}

\hints{
    \item Da \lstinline{containedPeople} leer initialisiert werden soll, brauchst du dafür keinen expliziten Aufruf in der Initialisierung.
    \item Um die Leute aussteigen zu lassen, die an ihrem Zielstockwerk angekommen sind, erstelle in der Methode zwei temporäre \lstinline{std::vector}-Container \lstinline{stay} und \lstinline{arrived}.
    Iteriere nun über alle Leute im Aufzug und prüfe, ob das Zielstockwerk der Person mit dem aktuellen Stockwerk des Aufzugs übereinstimmt.
    Wenn ja, lasse die Person aussteigen, indem du sie zu der \lstinline{arrived}-Liste mittels \lstinline{push_back()} hinzufügst.
    Andernfalls muss die Person im Aufzug verbleiben (\lstinline{stay}-Liste).
    Gib am Ende die arrived-Liste zurück, und ersetze \lstinline{containedPeople} durch \lstinline{stay}.
}

\subsection{Klasse Floor}
Implementiere die Klasse \lstinline{Floor}, die ein Stockwerk mit einer beliebigen Anzahl an wartenden Personen darstellt.

\lstinputlisting{problems/listings/sim1_floor.hpp}

\subsection{Klasse Building}
Schreibe eine Klasse \lstinline{Building}, die einen Aufzug besitzt der sich zwischen einer definierbaren Menge an Stockwerken bewegt und Personen befördert.

\lstinputlisting{problems/listings/sim1_building.hpp}

\subsection{Komfortfunktionen}
Erweitere die Klasse \lstinline{Building} um folgende \lstinline{public} Funktionen, um die Benutzung des Simulators von außen zu vereinfachen und lange Aufrufketten wie
\lstinputlisting{problems/listings/sim1_comfort.cpp}

zu vermeiden (\emph{Law of Demeter}\footnote{\url{https://en.wikipedia.org/wiki/Law_of_Demeter}}).
Der Simulator sollte nur mit Methoden der Klasse \lstinline{Building} kommunizieren.

\lstinputlisting{problems/listings/sim1_comfort_ex.cpp}

\subsection{Beförderungsstrategie}
Teste deine Implementation.
Erstelle dazu zunächst ein Gebäude und füge einige Personen hinzu.
%
\lstinputlisting{problems/listings/sim1_main.cpp}
%
Implementiere nun folgende Beförderungsstrategie.
Diese sehr einfache (und ineffiziente) Strategie fährt alle Stockwerke nacheinander ab, sammelt die Leute ein und befördert sie jeweils zu ihren Zielstockwerken.
%
\begin{algorithm}[H]
 \SetAlgoLined
 \For{Floor floor \textbf{in} Building}{
   Move \lstinline{elevator} to Floor \lstinline{floor};\\
   Let all people on \lstinline{floor} into \lstinline{elevator};\\
   
  \While{elevator has people} {
    Move Elevator to destination Floor of first Person in Elevator; \\
    Remove arrived people; \\
  }
 }
\end{algorithm}
%
\noindent Gib am Ende auch die verbrauchte Energie aus.
Schau dir die Ausgabe genau an und versuche nachzuvollziehen, warum Personen so oft kopiert werden.
Denke daran, dass diese bei einer Übergabe als Argument kopiert werden.

\hints {
    \item Falls du wie vorgeschlagen als Modell für den Energieverbrauch den Betrag der Differenz der abgefahrenen Stockwerke verwendest, solltest du am Ende \lstinline{consumedEnergy = 8} erhalten.
}
