%!TeX spellcheck = de_DE

\section{\ExercisePrefixBasics C++: Grundlagen, Funktionen und Strukturierung}
\label{sec:basics}
\cpppSolutionName{basics}{basics}
Für diese Aufgabe kannst du entweder das vorherige Programm weiterentwickeln oder genauso wie vorher ein neues Projekt anlegen.

\subsection*{Primitive Datentypen} 
Die primitiven Datentypen in C++ sind ähnlich zu denen in Java.
Allerdings sind alle Ganzzahl-Typen in C++ sowohl mit als auch ohne Vorzeichen verfügbar.
Standardmäßig sind Zahlen vorzeichenbehaftet.
Mittels \lstinline{unsigned} kann man vorzeichenlose Variablen deklarieren.
Durch das freie Vorzeichenbit kann ein größerer positiver Wertebereich dargestellt werden.

\cpppInputListing{01_basics/problems/listings/basics_types_intro.cpp}

Eine andere Besonderheit von C++ ist, dass Ganzzahlwerte implizit in Boolesche Werte (Typ: \lstinline{bool}) umgewandelt werden.
Werte, die ungleich 0 sind, werden als \lstinline{true} gewertet, 0 als \lstinline{false}.
Somit können Ganzzahlen direkt in Bedingungen ausgewertet werden.

\subsection{Größe von Datentypen}
Oft ist es notwendig die tatsächliche Größe der Datentypen im Speicher zu kennen. Diese kann mit Hilfe des \href{http://en.cppreference.com/w/c/language/sizeof}{\lstinline{sizeof}-Operators} ermittelt werden.
Deshalb sollst du dir in dieser Aufgabe die Größe der folgenden Datentypen in Bits, wie auch deren minimalen und maximalen Wert ausgeben lassen.

\cpppInputListing{01_basics/problems/listings/basics_types_ex.cpp}

\hints{

\item Die C++ Klasse \lstinline{std::numeric_limits}\footnote{\url{http://en.cppreference.com/w/cpp/types/numeric_limits}} bietet Funktionen sich minimale und maximale Werte von Datentypen ausgeben zu lassen. 
Einbinden lässt sich diese über den Header \lstinline{limits}.

}

\subsection{Sternenmuster mit Funktionen malen}

Schreibe eine Funktion \lstinline{printStars(int n)}, die \lstinline{n}-mal einen Stern (*) auf die Konsole ausgibt und mit einem Zeilenumbruch abschließt.
Ein Aufruf von \lstinline{printStars(5)} sollte folgende Ausgabe generieren:

\cpppInputListing{01_basics/problems/listings/basics_stars.cpp}

Platziere die Funktion \textbf{vor} \lstinline{main}, da sie ansonsten von dort aus nicht aufgerufen werden kann.
Erstelle nun eine zweite Funktion \lstinline{printFigure(int n)}, die \lstinline{printStars(int n)} nutzt um eine Figur wie unten dargestellt auszugeben.
Verwende hierzu eine Schleife.

\cpppInputNoPageBreakListing{01_basics/problems/listings/basics_starpattern.cpp}

\hints{
    \item Was die Benennung von Funktionen, Variablen und Klassen angeht, bist du frei. Für Klassen ist \enquote{CamelCase} wie in Java üblich. Bei Funktionen und Variablen wird zumeist entweder auch CamelCase oder Kleinschreibung mit Unterstrichen verwendet.
    \item Um Strings auszugeben, stellt dir C++ den Stream \lstinline{std::cout} zur Verfügung, welches den String auf die Standardausgabe ausgiebt. Einen Zeilenumbruch erreicht man, durch anfügen von \lstinline{std::endl}.
}

\subsection{Erweiterung durch einen enum-Typen}
\label{sec:basics_enumtype}
Ein Aufzählungstyp (engl. enumerated type) ist ein Datentyp, dessen Wert auf eine definierte Menge begrenzt ist.
Alle möglichen Werte werden bereits bei der Definition des Typs mit einem eindeutigen Namen angegeben. Um einen solchen Datentyp zu definieren, benutzt man das Schlüsselwort \lstinline{enum}: 

\cpppInputListing{01_basics/problems/listings/basics_enum_example.cpp}

Die Werte in den geschweiften Klammern sind Symbole, die intern als Zahlen (Integer) gespeichert sind. Die Werte beginnen normalerweise bei 0, man kann sie aber auch selbst festlegen:

\cpppInputListing{01_basics/problems/listings/basics_enum_example_2.cpp}

Nutzen kann man den Enum-Typ nun, indem man eine Variable von diesem Typ deklariert und diese wie gewohnt verwendet:

\cpppInputListing{01_basics/problems/listings/basics_enum_example_3.cpp}

Deine Aufgabe ist es jetzt, das Programm um einen Enum-Typen \lstinline{Direction} zu erweitern, der die Richtung (\lstinline{Left}, \lstinline{Right}) angibt, in die das Pattern ausgegeben werden soll.
Erweitere dazu, falls nötig, die Funktionen \lstinline{printStars} bzw. \lstinline{printFigure}. Es könnte außerdem nützlich sein, eine Funktion \lstinline{printSpaces} einzuführen.
Das Resultat soll am Ende jeweils folgendermaßen aussehen:

\cpppInputNoPageBreakListing{01_basics/problems/listings/basics_starpattern_alignment.cpp} 

\subsection{Auslagern der Datei}
Erstelle eine neue Header-Datei \textbf{\filename{functions.hpp}} und eine neue
Sourcedatei \textbf{\filename{functions.cpp}}.
Klicke hierzu mit der \textbf{rechten Maustaste} auf den Ordner \textbf{\filename{src}} und wähle \textbf{Add a New File}, wähle den Dateitypen \texttt{Header File} und gebe der Datei den Namen \filename{functions}. 
Bestätige den Dialog mit \textbf{OK}.
Wiederhole diesen Schritt entsprechend für \texttt{C++ Source File} und ebenfalls dem Namen \filename{functions}.
Füge in der Headerdatei die folgenden Include Guards hinzu.

\cpppInputListing{01_basics/problems/listings/basics_hpp_guards.hpp}

Binde danach \filename{functions.hpp} in beide Sourcedateien (\filename{functions.cpp} und \filename{main.cpp}) ein, indem du

\cpppInputListing{01_basics/problems/listings/basics_hpp_include.hpp}

verwendest.
\textbf{Verschiebe} die beiden zuvor erstellten Funktionen nach \filename{functions.cpp}.

Schreibe nun in \filename{functions.hpp} \textbf{Funktionsprototypen} für die Funktionen, die sich nun in \filename{functions.cpp} befinden.
Funktionsprototypen dienen dazu, dem Compiler mitzuteilen, dass eine Funktion mit bestimmtem Namen, Parametern und Rückgabewert existiert.
Ein Prototyp ist eine mit \textbf{;} abgeschlossene Signatur der Funktion ohne Funktionsrumpf.
Der Prototyp von \lstinline{printStars(int n)} lautet 

\cpppInputListing{01_basics/problems/listings/basics_printstars_sig.hpp}

Auch der in der vorherigen Aufgabe erstellte Enum-Typ sollte in der Header-Datei platziert werden.

Fertig -- die Ausgabe des Programms sollte sich nicht verändert haben.

\hints{
    \item Sourcedateien tragen in der Regel die Endung \filename{.cpp}, Headerdateien \filename{.h} oder \filename{.hpp}.
    \item Denke daran, auch in \filename{functions.cpp} den Header \lstinline{iostream} einzubinden, falls du dort Ein- und Ausgaben verwenden willst (\lstinline{#include<iostream>}).
    \item Beachte, dass es zwei verschiedene Möglichkeiten gibt, eine Header-Datei einzubinden - per \lstinline{\#include <Bibliotheksname>} sowie per \lstinline{\#include "Dateiname"}.
    Bei der ersten Variante sucht der Compiler nur in den Include-Verzeichnissen der Compiler-Toolchain, während bei der zweiten Variante die Projektordner durchsucht werden. Somit eignet sich die erste Schreibweise für System-Header und die zweite für projektspezifische Header.
    \item Anstelle der Include Guards kannst du auch die Präprozessordirektive \lstinline{\#pragma once} verwenden.
    Diese ist zwar nicht standardisiert, wird aber von den meisten Compilern unterstützt.
}

\subsection{Dokumentation \optional} \label{basics:doc}

\optionaltextbox

Für die Lesbarkeit eines Programms ist eine ausführliche Dokumentation des Programmcodes essentiell.
Dazu wirst du in dieser Aufgabe den Header \filename{functions.hpp} mit Kommentaren versehen, die das Tool \emph{Doxygen}\footnote{Doxygen-Projektseite: \url{http://www.doxygen.nl/}} interpretieren kann.

Damit Doxygen deine Kommentare erkennt, muss ein spezielles Format eingehalten werden.
\begin{itemize}
    \item Kommentare müssen vor den jeweiligen zu kommentierenden Elementen (\zB Funktionen, Klassen) stehen.
    \item{
    Mehrzeilige Kommentare müssen den folgenden Stil einhalten und mit einem Doppelstern beginnen (\lstinline{/**}):
    
	\cpppInputListing{01_basics/problems/listings/basics_doxygen_comment.cpp}
    }
\end{itemize}

Außerdem müssen bestimmte Befehle in den Kommentaren verwendet werden, die Doxygen bei der Dokumentationsgenerierung verwenden kann.
Eine Liste aller Doxygen-Befehle findest du bei Interesse unter \url{http://www.stack.nl/~dimitri/doxygen/manual/commands.html}.
\footnote{Man kann seinen Kommentar auch speziell formatieren und so diese Kommandos teilweise weglassen.
Beispiele unter \url{http://www.stack.nl/~dimitri/doxygen/manual/docblocks.html\#docexamples}}.
Diese Kommandos sind die folgenden:

\medskip
    \begin{tabular}{ll}
        {\lstinline!@file Dateiname!} & Damit Doxygen das komplette File parst. \\
        {\lstinline!@brief KurzeBeschreibung!} & Einzeilige Beschreibung des zu dokumentierenden Elements. \\
        {\lstinline!@author AutorenName!} & Name des Autors des zu dokumentierenden Elements. \\
        {\lstinline!@param Parametername Beschreibung!} & Je Funktionsparameter eine Zeile, die seinen Zweck erläutert. \\
        {\lstinline!@return Beschreibung!} & Kurze Beschreibung der Rückgabe. \\
    \end{tabular}
\medskip

Da das Dokumentieren der Datei nicht vor der Datei passieren kann (wo sollte das sein?), geschieht es deshalb direkt nach den Präprozessor-Direktiven.

Deine Aufgabe ist es nun, den von dir erstellten Code sorgfältig zu dokumentieren.
Die Dokumentation geschieht dabei in der \filename{.h} oder \filename{.hpp} Datei. 
Hier ein kleines Beispiel dazu, damit du eine Vorstellung davon bekommst, wie das ganze am Ende auszusehen hat.

\cpppInputListing{01_basics/problems/listings/basics_doxygen_example.hpp}

\paragraph{Erstellen der HTML-Dokumentation}
Erstellen kannst du die Dokumentation am Ende über die Kommandozeile.
Dafür öffnest du das Terminal mit \lstinline{Ctrl + Shift + T} und wechselst in das Verzeichnis, in dem dein Projekt liegt (üblicherweise \lstinline{cd ~/CPPP-Workspace/NameDeinesProjektes}).
Dort gibst du \lstinline{doxygen -g} ein, was dir eine vorgefertigte Konfigurationsdatei für Doxygen generiert. 
Mit dem Befehl \lstinline{doxygen} kannst du dir jetzt die fertige Dokumentation generieren lassen.
Diese befindet sich nun im Verzeichnis \filename{html} in der Datei \filename{index.html}.
Öffnest du diese Datei per Doppelklick, findest du unter dem Menüpunkt \lstinline{Files} die von dir dokumentierten Dateien.

\subsection{Eingabe}
\paragraph{Eingabe der Breite}
Erweitere das Programm um eine Eingabeaufforderung zur Bestimmung der Breite des auszugebenden Musters.
Die Breite soll dabei eine im Programmcode vorgegebene maximale Breite (\zB \numprint{80} Zeichern) nicht überschreiten dürfen. 
Gib gegebenenfalls eine Fehlermeldung aus und frage den Benutzer erneut nach der Breite des Musters.
Verwende zum Einlesen \lstinline{std::cin} und \lstinline{operator>>} wie in folgendem Beispiel.

  \cpppInputListing{01_basics/problems/listings/basics_cin.cpp}
  
\paragraph{Eingabe der Richtung}
In \ref{sec:basics_enumtype} hast du einen enum-Typen definiert, der die Richtungen definiert, in die das Muster ausgegeben wird. 
Füge nun eine weitere Eingabeaufforderung hinzu, die vom Nutzer die gewünschte Richtung erfragt (beispielsweise 0 für \lstinline{Left} und 1 für \lstinline{Right}). 
Gibt der Nutzer eine ungültige Richtung ein, kannst du eine Fehlermeldung ausgeben oder erneut die Eingabe abfragen.\\
\\
Erstelle auch für diesen Aufgabenteil eine eigene Funktion in \filename{functions.cpp}.

\subsection{Fortlaufendes Alphabet ausgeben}
Statt eines einzelnen Zeichens soll nun das fortlaufende Alphabet ausgegeben werden.
Sobald das Ende des Alphabets erreicht wurde, beginnt die Ausgabe erneut bei \texttt{a}.
Beispiel:

  \cpppInputListing{01_basics/problems/listings/basics_charpattern.cpp}

Implementiere dazu eine Funktion \lstinline{char nextChar()}.
Diese soll bei jedem Aufruf das nächste auszugebende Zeichen vom Typ \lstinline{char} zurückgeben, beginnend bei 'a'.
Dazu muss sich \lstinline{nextChar()} intern das aktuelle Zeichen merken.
Dies kann durch die Verwendung von statischen Variablen erreicht werden. 
Diese behalten ihren aktuellen Wert auch nach Verlassen der Funktion. 
Das heißt, wenn \lstinline{nextChar()} das nächste Mal aufgerufen wird, steht der Wert des vorherigen Zeichens noch zur Verfügung.
Eine statische Variable \lstinline{c} wird wie folgt deklariert:

  \cpppInputListing{01_basics/problems/listings/basics_staticchar.cpp}

In diesem Fall wird die Variable \lstinline{c} \textbf{einmalig beim ersten Aufruf} mit \lstinline{'a'} initialisiert und kann später beliebig verändert werden.

\hints{
\item Der Datentyp \lstinline{char} kann wie eine Zahl verwendet werden, \dasheisst man kann die Modulooperation \lstinline{\%} verwenden um am Ende des Alphabets wieder zu \lstinline{'a'} umzubrechen.
}

\subsection{Namensräume}
Ein Namensraum (engl. namespace) dient dazu, Namenskonflikte zu vermeiden.
Erweitere dazu das Programm, indem du im Header die Funktionsprototypen wie
folgt in einen \lstinline{namespace} setzt. Achte auch auf das Semikolon nach der schließenden Klammer.

  \cpppInputListing{01_basics/problems/listings/basics_namespace_fun.cpp}

Denke daran, dass du die Namen der Funktionen in der Sourcedatei noch anpassen musst, indem du vor jede Funktion den gewählten \lstinline{namespace}-Namen gefolgt von zwei Doppelpunkten setzt.
Genauso muss der Namensraum auch vor jeden Aufruf der Funktion gesetzt werden.

  \cpppInputListing{01_basics/problems/listings/basics_namespace_ref.cpp}

Vergisst man, den Namensraum in der Sourcedatei vor den Funktionsaufrufen anzugeben, findet der Linker keine Implementation zu der im Header definierten Funktion.
Weiterhin stünde diese Funktion nicht mehr im Bezug zum Header und könnte nur noch lokal verwendet werden (\lstinline{print_star(int n)} und \lstinline{fun::print_star(int n)} sind zwei unterschiedliche Funktionen!). \\

Falls man seine Funktionen noch weiter unterteilen möchte, kann man Namensräume auch schachteln.
Hierzu definiert man wie oben einen weiteren Namensraum mit \lstinline{namespace} in einer bereits vorhandenen Namensraum-Instanz. Wichtig ist auch hier wieder das Semikolon nach jeder schließenden Klammer der Namespace-Definition.

  \cpppInputNoPageBreakListing{01_basics/problems/listings/basics_namespace_nested.cpp}

In der Sourcedatei folgt dann nach dem ersten Namensraum (hier \lstinline{fun}) der geschachtelte Namensraum \lstinline{ny}. Die verschiedenen Ebenen der Namensräume werden durch zwei Doppelpunkte (den sogenannten Scope-Resolution-Operator) voneinander getrennt.
Danach können die Funktionen in dem Namensraum \lstinline{ny} verwendet werden, indem man diese ebenfalls mit dem Scope-Resolution-Operator an die Namensraum-Hierarchie anhängt.

\cpppInputListing{01_basics/problems/listings/basics_namespace_nested_ref.cpp}

In diesem Projekt wird dies nicht notwendig sein, da die Anzahl der definierten Funktionen überschaubar ist, aber trotzdem empfehlen wir dir, es auszuprobieren.

\hints{
    \item Du kannst \lstinline{using namespace fun;} verwenden, um diesen Namensraum zu importieren (vergleichbar mit \lstinline{static import} in Java).
    Allerdings kann es dabei leichter zu Namenskollisionen zwischen Elementen der verschiedenen Namensräume kommen. Deshalb sollte der Befehl \lstinline{using namespace} mit Bedacht verwendet werden.
    \item Bei dem geschachtelten Namensraum ist entsprechend \lstinline{using namespace fun::ny;} zu verwenden um den Namensraum zu importieren.
}
