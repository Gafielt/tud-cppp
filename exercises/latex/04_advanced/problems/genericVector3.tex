\section{\ExercisePrefixAdvanced Generische Vektor-Implementation (Templates)}
\label{sec:genericVector}
Erinnere dich an die Klasse \lstinline{Vector3} aus dem ersten Praktikumstag (\ref{sec:overloading}).
Diese hat den Datentyp \lstinline{double} für die einzelnen Komponenten verwendet.
Schreibe die Klasse so um, dass der Datentyp der Komponenten durch einen Template-Parameter angegeben werden kann.
Füge dafür der Klasse \lstinline{Vector3} einen Template-Parameter hinzu und ersetze jedes Aufkommen von \lstinline{double} mit dem Template-Parameter.
Vergiss nicht, die Implementation in den Header zu verschieben, da der Compiler die Definition einer Klasse kennen muss, um beim Einsetzen des Template-Parameters den richtigen Code zu generieren.

Verbessere außerdem die Effizienz und Sauberkeit der \lstinline{Vector3}-Klasse, in dem du die Parameterübergabe in den entsprechenden Methoden auf \lstinline{const} Referenzen umstellst und alle Getter als \lstinline{const} deklarierst.

Du weißt bereits, dass alle \lstinline{template}-Funktionen und -Methoden im Header enthalten sein müssen.
Um den Code trotzdem zu strukturieren, hat es sich eingebürgert, dass man die Klassendefinition in der \filename{hpp}-Datei hält, ohne die Methoden zu implementieren.
Im Anschluss wird eine \filename{tpp}-Datei inkludiert, die die Implementierung der Methoden und Funktionen enthält.
%
Der Aufbau der Datei \filename{Vector3.hpp} wäre also wie folgt:
\cpppInputListing{04_advanced/problems/listings/genericVector3_tpp_include.hpp}

\hints{
    \item
    Die Datei \filename{Vector3.tpp} ist nicht vorgegeben, du musst diese selbst erstellen!
    \item
    Auch wenn bei reinen Template-Klassen die \filename{cpp}-Datei leer bleibt, ist es sinnvoll, eine solche anzulegen.
    Dadurch wird das Template garantiert auf Syntaxfehler überprüft.
    Der Inhalt der \filename{cpp}-Datei ist in dieser Aufgabe schlicht \lstinline{#include "Vector3.hpp"}.
}
