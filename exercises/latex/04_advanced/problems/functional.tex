\section{\ExercisePrefixAdvanced Funktionales Programmieren} %\experimental}
\label{sec:functional}
%\experimentaltextbox

In dieser Aufgabe werden Funktionen aus der funktionalen Programmierung vorgestellt.
Diese sind \lstinline{map}, \lstinline{filter} und \lstinline{reduce}.
%
Der Ablauf ist wie folgt:
\begin{itemize}
\item
In \ref{sec:map-filter-reduce-intro} werden erst einmal die Funktionsweisen der zu implementierenden Funktionen \lstinline{map}, \lstinline{filter} und \lstinline{reduce} vorgestellt.
\item
In \ref{sec:map-filter-reduce-basic-impl} wirst du diese Funktionen und zusätzliche Hilfsfunktionen implementieren.
\item
In \ref{sec:functional_functor} wirst du deine Hilfsfunktionen als \emph{Funktoren} implementieren und \lstinline{map}, \lstinline{filter} und \lstinline{reduce} entsprechend anpassen.
\item
In \ref{sec:functional_template} wirst du den Code auf Templates umstellen, damit Funktionen und Funktoren austauschbar verwendet werden können.
\item
In \ref{sec:functional_method} wirst du zusätzlich mit Methodenzeigern arbeiten.
\end{itemize}

\subsection{Erklärung \lstinline{map}, \lstinline{filter} und \lstinline{reduce}}
\label{sec:map-filter-reduce-intro}

Arbeitet man auf iterierbaren Sequenzen, ist dies fast immer mit Schleifen über die Sequenz verbunden.
Die drei Funktionen \lstinline{map}, \lstinline{filter} und \lstinline{reduce} vereinfachen uns hierbei die Arbeit.
Hierzu ein Beispiel:
Haben wir einen Vektor des Typs \lstinline{double} und wollen jedes Element quadrieren, endet dies meist in dem folgenden Programmcode:

\cpppInputListing{04_advanced/problems/listings/functional_intro_example.cpp}

\paragraph{map}
Die Idee von der Funktion \lstinline{map} ist es, genau dies zu vereinfachen.
Die Funktion erhält folgende Parameter: 
\begin{itemize}
\item
Die Start- und Enditeratoren der zu modifizierenden Sequenz
\item
Einen Iterator, der auf eine Sequenz zeigt, in der die veränderten Elemente gespeichert werden sollen
\item
Einen Funktionszeiger, der auf eine Funktion zeigt, die für jedes Element der iterierbaren Sequenz aufgerufen werden soll
\end{itemize}

Ein Beispiel siehst du in folgendem Listing:

\cpppInputListing{04_advanced/problems/listings/functional_map_example.cpp}

\paragraph{filter}
Die Funktion \lstinline{filter} funktioniert analog, indem sie einen Zeiger auf eine Funktion erhält, die einen Listenelementtyp erwartet und ein Booleschen Wert (\lstinline{bool}) zurückgibt.
Auf alle Elemente wird diese Funktion aufgerufen und alle Elemente, für die die Funktion \lstinline{true} zurückgibt, werden in die Ausgabesequenz kopiert. Der Rest wird entfernt. \\
%
\cpppInputListing{04_advanced/problems/listings/functional_filter_example.cpp}
%
Die Besonderheit hier ist, dass wir eine zweite Liste (\lstinline|filteredNumbers|) benötigen, um das Ergebnis zu speichern, da die Ergebnisliste in der Regel kürzer sein wird als die Eingabeliste.
Eine andere Lösung wäre, den letzten Stand des Ausgabeiterators zurückzugeben und die Liste entsprechend einzukürzen.

Zusätzlich zur Ausgabeliste verwenden wir hier einen \lstinline|std::back_insert_iterator|\footnote{\url{http://en.cppreference.com/w/cpp/iterator/back_insert_iterator}}, der die ihm geordnete Liste immer dann vergrößert, wenn ein neues Element in den Iterator hineingeschrieben wird.
Die Hilfsfunktion \lstinline|std::back_inserter|\footnote{\url{http://en.cppreference.com/w/cpp/iterator/back_inserter}} vereinfacht die Erzeugung des Iterators.
Sowohl die Klasse \lstinline|std::back_insert_iterator| als auch die Hilfsfunktion \lstinline|std::back_inserter| befinden sich im Header \lstinline|iterator|.

\paragraph{reduce}
Die Aufgabe der Funktion \lstinline{reduce} ist es, eine Sequenz zu einem einzelnen Element zusammenzuschrumpfen.
Hierbei wird der Ausgabeiterator gegen einen Startwert ausgetauscht.
Hier ein Beispiel, bei dem die Summe über die Elemente in \lstinline{numbers} gebildet wird.
%
\cpppInputListing{04_advanced/problems/listings/functional_reduce_example.cpp}

\subsection{Programmieren der Funktionen}
\label{sec:map-filter-reduce-basic-impl}

Du wirst nun die drei Funktionen \lstinline{map}, \lstinline{filter} und \lstinline{reduce} nachprogrammieren.
Hierbei geht es erstmal darum, ein funktionierendes Gerüst zu erstellen, anstatt perfekt generische Algorithmen zu erhalten.

\subsubsection{\lstinline{map}}

Schreibe eine Funktion \lstinline{map} die folgende Signatur besitzt.

\cpppInputListing{04_advanced/problems/listings/functional_map_sig.cpp}

Hierbei ist der letzte Parameter der Funktionszeiger.
Die Klammern um \lstinline{*func} sind notwendig, damit der Compiler den übergebenen Parameter als Funktionszeiger einer Funktion mit Rückgabewert \lstinline{double} interpretiert und nicht als Funktion mit Rückgabewert \lstinline{double *}\footnote{Welche folgende Signatur hätte: \lstinline{double *(*func)(double d)}}.
Diese Funktion hat zusätzlich noch ein \lstinline{double d} als Parameter.

Du kannst dich bei der Implementierung von \lstinline|map| an dem Schleifengerüst zu Anfang von \ref{sec:map-filter-reduce-intro} orientieren.

\subsubsection{\lstinline{filter}}

Die von dir zu schreibende Funktion \lstinline{filter} soll der folgenden Signatur folgen.
%
\cpppInputListing{04_advanced/problems/listings/functional_filter_sig.cpp}
%
Die Implementierung von \lstinline|filter| wird sehr ähnlich zur Implementierung von \lstinline|map| aussehen.
Der Hauptunterschied ist, dass \lstinline|pred| in einer \lstinline|if|-Bedingung eingesetzt werden muss, um zu entscheiden, ob der aktuelle Wert in den Ausgabeiterator (\lstinline|out_first|) geschrieben werden soll.

\subsubsection{\lstinline{reduce}}

Erstelle eine Funktion \lstinline{reduce}, die der folgenden Signatur folgt.
%
\cpppInputListing{04_advanced/problems/listings/functional_reduce_sig.cpp}
%
Hierbei muss ein passender initialer Wert übergeben werden, der mit dem Rückgabewert und dem ersten Argument der übergebenen Funktion zusammenpasst (\lstinline|typename RetT|).
Der zweite Parameter der Funktion kann sich im Typ sogar unterscheiden (\lstinline|typename RHS|).

\subsubsection{Hilfsfunktionen implementieren}

Implementiere in dieser Aufgabe drei Hilfsfunktionen, die den Anforderungen der jeweiligen Signaturen der Funktionszeiger in den Funktionen \lstinline{map}, \lstinline{filter} und \lstinline{reduce} folgen.
Du kannst dir dabei gerne eigene Funktionen ausdenken oder dich an die Funktionen in den Beispielen halten (\bspw \lstinline|square| für \lstinline|map|, \lstinline|isOdd| für \lstinline|filter|, \lstinline|sum| für \lstinline|reduce|). 

Teste anschließend deine Implementierungen mithilfe deiner Hilfsfunktionen.

\subsection{Funktoren}
\label{sec:functional_functor}
Es gibt außerdem noch die Möglichkeit, Funktionen in einem Funktionsobjekt (\emph{Funktor}) zu kapseln.
Dabei überlädt man den Operator \lstinline{operator()} der Funktor-Klasse, welcher eine bestimmte Funktion ausführt. 
Schaut man sich in unserem Beispiel die Funktion \lstinline{square} mit der Definition \lstinline{double square(double i);} an, würde der Funktor folgendermaßen aussehen:
%
\cpppInputListing{04_advanced/problems/listings/functional_functor_square.cpp}
%
Die Funktion \lstinline{map} würde wie folgt umgeschrieben werden müssen, um den Funktor zu akzeptieren:
%
\cpppInputListing{04_advanced/problems/listings/functional_functor_map.cpp}
%
Erstelle für jede deiner Hilfsfunktionen eine Funktor-Klasse und füge neue Implementierungen für \lstinline{map}, \lstinline{filter} und \lstinline{reduce} hinzu, die mit den Funktoren kompatibel sind.
Vergleiche die Ausgabe der Funktionszeiger- und Funktoren-basierten Implementierungen.

\subsection{Verwendung von Templates}
\label{sec:functional_template}
Die derzeitige Implementierung funktioniert entweder mit Funktoren einer bestimmten Klasse oder mit Funktionszeigern, die einem bestimmten Typen angehören, der durch die Signatur der Funktion festgelegt ist.
Diese Verdoppelung des Codes ist unschön und 
Um das Problem zu lösen, kann man die Funktionszeiger-/Funktorvariable durch einen Templateparameter ersetzen.
Damit ist der Parametertyp flexibel;
der Nachteil ist, dass nur noch aus der eigentlichen Implementierung der Funktion hervorgeht, was der Typ des Templateparameters anbieten muss (\emph{Induzierte Schnittstelle}).

Erstelle Varianten der Funktionen \lstinline{map}, \lstinline{filter} und \lstinline{reduce}, deinen weiteren Templateparameter angeben, der flexibel mit Funktionszeigern oder Funktoren belegt werden kann.
%
\cpppInputListing{04_advanced/problems/listings/functional_template_typename.cpp}
%
Vergleiche die Ausgabe deiner Template-basierten Lösung mit den Ergebnissen der Funktionszeiger- und Funktor-basierten Lösungen.

\subsection{Methodenzeiger}
\label{sec:functional_method}
Es gibt auch die Möglichkeit, Methoden\footnote{Hier ist eine einfache Erklärung zu dem Unterschied von Funktion und Methode zu finden \url{http://stackoverflow.com/a/155655}} via Zeiger auszuführen (sogenannte \textbf{Methodenzeiger}).
Hierzu muss man zusätzlich zu dem Funktionsnamen noch ein Objekt übergeben, auf das die Methode angewendet wird.

In unserem Beispiel von \lstinline{Square} fügen wir noch eine weitere Methode \lstinline{squareroot} hinzu, welche das Inverse der Quadrierung ausführt.
Unsere Klasse Square verändert sich dementsprechend zu
%
\cpppInputListing{04_advanced/problems/listings/functional_method_square.cpp}
%
und unser \lstinline{map} zu 
%
\cpppInputListing{04_advanced/problems/listings/functional_method_map.cpp}
%
Deine Aufgabe ist es nun, eine neue Implementation von \lstinline{map} hinzuzufügen, deinem für \lstinline{map} geschriebenen Funktor eine weitere Methode hinzuzufügen und anschließend deine Implementation mit allen implementierten Methoden zu testen.

\subsection*{Nachwort zu dieser Aufgabe}
Für produktive C++-Programme bietet die Standardbibliothek fertige Funktionen und Klassen, um die gerade erlernten Prinzipien dieser Aufgabe zu realisieren, z.B. \lstinline{std::function<...>}\footnote{\url{http://en.cppreference.com/w/cpp/utility/functional/function}} und \lstinline{std::bind()}\footnote{\url{http://en.cppreference.com/w/cpp/utility/functional/bind}}.
Diese können mit einer beliebigen Anzahl von Parametern umgehen und beinhalten viele weitere Features.
