\section{\ExercisePrefixAdvanced Makefiles}

In dieser Übung erkunden wir, wie man Makefile-Projekte in CodeLite aufbaut.
Wir bauen dafür einen Teil unseres Aufzugsimulators nach, um zu sehen, wo die Herausforderungen in echten Projekten mit Makefiles gelöst werden können.

Das Programm \emph{make}\footnote{Dokumentation \url{https://www.gnu.org/software/make/manual/make.html}} wird in großen Softwareprojekten verwendet, um Programme zu kompilieren.
\texttt{Makefiles} geben \texttt{make} dabei Informationen wie das Programm gelinkt und compiliert werden muss.
Außerdem kann es verwendet werden, um weitere Nebenaufgaben während der verschiedenen Kompilationsphasen zu definieren, wie zum Beispiel das automatische Löschen der später unnötigen Kompilationsdateien.
All das werden wir in dieser Aufgabe ausprobieren.

\subsection{Projekt anlegen:}
Wähle \emph{Workspace $\to$ New project} und wähle als Projekttyp \textbf{CPPP/C++ Projekt}.

Das erzeugte Projekt enthält bereits eine Datei \texttt{Makefile} (im Ordner \texttt{resources}) und eine \texttt{main.cpp}.

In dieser Übung wollen wir unser eigenes Makefile erstellen. Öffne dafür die Datei \texttt{Makefile} und lösche ihren Inhalt. 

\texttt{Make} erwartet, dass das Makefile ein Target mit dem Namen \lstinline{all} hat.
Um unser Projekt zu testen, geben wir zunächst eine einfache Meldung auf der Kommandozeile aus.
Lege nun das Target \lstinline{all} an und füge den folgenden Befehl an (Vergiss dabei nicht den Tab vor jedem Befehl!):

\begin{lstmake}
all:
    @echo "Running all..."
\end{lstmake}

Wenn du jetzt \emph{Build} aufrufst, sollte in der Konsole in etwa Folgendes erscheinen:
\begin{verbatim}
----------Building project:[ mktest - Debug ]----------
Running all...
====0 errors, 0 warnings====

\end{verbatim}

\subsection{Erster Kompiliervorgang}
Jetzt ist es an der Zeit, ein Programm mittels \texttt{make} zu kompilieren.
Lege dazu eine C++-Sourcedatei \filename{main.cpp} mit einer \lstinline{main}-Funktion an, die etwas sinnvolles ausgibt.

Entgegen unserer bisherigen Erfahrung musst du nun manuell im \filename{Makefile} eintragen, dass \filename{main.cpp} gebaut werden soll.
Ersetze die Dummy-Ausgabe daher durch einen Compiler-Aufruf an \texttt{g++}:
\begin{lstmake}
all:
    g++ -o main.exe main.cpp
\end{lstmake}

Wenn du jetzt \emph{Build} aufrufst, wird dein Programm kompiliert und als \filename{main.exe} im Projekthauptverzeichnis abgelegt.

\hints {
    \item
    Die Option $-o$ für g++ erwartet als ersten Parameter den Namen der gewünschten Zieldatei.
}

\subsection{Klasse \lstinline{Building}:}
Jetzt fügen wir die Klasse \lstinline{Building} zu unserem Projekt hinzu, die allerdings nur minimale Funktionalität bietet.

Building.hpp:
\begin{minipage}[t]{.45\textwidth}
\cpppInputListing{06_elevator/problems/listings/sim4_building.hpp}
\end{minipage}

Building.cpp:
\begin{minipage}[t]{.5\textwidth}
\cpppInputListing{06_elevator/problems/listings/sim4_building.cpp}
\end{minipage}

Erzeuge in der \lstinline{main}-Funktion eine zweistöckige Instanz von \lstinline{Building} und gib diese mittels \lstinline{Building::toString} auf der Konsole aus.
Damit das Projekt kompiliert, muss auch \lstinline{Building} im Makefile eingetragen werden.
Passe dazu den Kompileraufruf an:
\begin{lstmake}
all:
    g++ -o main.exe main.cpp Building.cpp
\end{lstmake}

Wenn du das Projekt gebaut hast und ausführst, sollte auf der Konsole eine Ausgabe deines Gebäudes erscheinen.

\subsection{Compiler-Aufrufe auslagern:}
In der Vorlesung haben wir gesehen, dass \emph{make} anhand der Zeitstempel von Dateien dazu in der Lage ist, zu erkennen, wann ein Programmteil neu gebaut werden muss.
Aktuell nutzen wir diese Möglichkeit noch nicht:
Egal ob wir \filename{main.cpp}, \filename{Building.cpp} oder \filename{Building.h} verändert haben, immer wird das gesamte Projekt neu gebaut.
In diesem Schritt zerlegen wir die Abhängigkeiten zu den einzelnen Dateien.

Mache das Target \texttt{all} jetzt abhängig von den Objektdateien \filename{main.o} und \filename{Building.o} und erzeuge für jede Objektdatei ein eigenes Ziel, welches diese baut (Das Flag \texttt{-c} sorgt dafür, dass die Sourcedateien nur kompiliert, aber nicht gelinkt werden).

\begin{lstmake}
all: main.o Building.o
    g++ -o main.exe main.o Building.o

main.o: main.cpp
    g++ -c -o main.o main.cpp

Building.o: Building.cpp
    g++ -c -o Building.o Building.cpp
\end{lstmake}

Baue das Projekt nun erneut, du solltest drei Aufrufe von \texttt{g++} sehen:

\begin{minipage}{\textwidth}
\begin{lstmake}
make all
g++ -c -o main.o main.cpp
g++ -c -o Building.o Building.cpp
g++ -o main.exe main.o Building.o
\end{lstmake}
\end{minipage}

Baust du das Projekt nun ein weiteres Mal, so wird nur noch der Linker aufgerufen:
\begin{verbatim}
make all
g++ -o main.exe main.o Building.o
\end{verbatim}

\subsection{Linker-Aufruf auslagern}
Wie können wir diesen an sich unnötigen Aufruf ebenfalls noch loswerden?
Eine Lösung ist es, das Target \texttt{all} von \filename{main.exe} abhängig zu machen und ein neues Ziel \filename{main.exe} zu definieren:
\begin{lstmake}
all: main.exe

main.exe: main.o Building.o
    g++ -o main.exe main.o Building.o

# ...
\end{lstmake}

Wenn du das Projekt jetzt baust, erhältst du erfreulicherweise die Rückmeldung, dass nichts zu tun ist:
\begin{verbatim}
make all
make: Nothing to be done for `all'.
\end{verbatim}

\subsection{Inkrementelles Bauen:}
Wir erproben jetzt, wie sich Veränderungen an einer der drei Dateien auf die Ausführung von \texttt{make} auswirken.
Mache nacheinander kleine Änderungen -- das können auch Kommentare sein -- an den Dateien \filename{main.cpp}, \filename{Building.h} und \filename{Building.cpp} und baue das Projekt nach jeder Änderung.

Dir fällt auf, dass Änderungen an \filename{Building.h} von \texttt{make} nicht bemerkt werden; die Datei taucht ja nirgendwo explizit im Makefile auf.

Wir sehen uns jetzt an, welche Tragweite dieses Problem haben kann.

\subsection{Header als Abhängigkeiten}
Du hast kennengelernt, dass man Implementationen auch \texttt{inline} in einem Header machen kann, zum Beispiel wenn diese klein sind.

Bewege \lstinline{toString} nun nach \filename{Buidling.h}:
\cpppInputListing{06_elevator/problems/listings/sim4_tostring.cpp}

Baue das Projekt; es kompiliert nicht! Warum? Genau aus dem Grund, dass \texttt{make} das Header-File nicht \glqq kennt\grqq{}.
Jetzt gibt es im Projekt keine Definition von \lstinline{toString} wie uns der Linker auch mitteilt:

\begin{minipage}{\textwidth}
\begin{verbatim}
main.o:main.cpp:(.text+0x5c): undefined reference to `Building::toString() const'
collect2: ld returned 1 exit status
\end{verbatim}
\end{minipage}

Das Problem lässt sich lösen, indem wir im Makefile angeben, dass \filename{main.o} nicht nur abhängig von \filename{Building.cpp}, sondern auch von \filename{Building.h} ist:
\begin{lstmake}
# ...
main.o: main.cpp Building.h
    g++ -c -o main.o main.cpp
# ...
\end{lstmake}

Ist das eine schöne Lösung?
Sicherlich nicht, denn ab sofort müssten wir manuell alle Header ins Makefile eintragen, die wir per \lstinline{#include} in eine Sourcedatei einbinden.
Schlimmer noch: Wir müssten über rekursive Inkludierungen Bescheid wissen, z.B. wenn \filename{Building.h} einen anderen veränderlichen Header wie \filename{Floor.h} einbindet.

Glücklicherweise hilft uns \texttt{g++} bei diesem Problem.

\subsection{Header automatisch als Abhängigkeiten deklarieren:}

Wir automatisieren jetzt die Erkennung von Headern als Abhängigkeiten.
Lösche dazu die Abhängigkeit \filename{Building.h} des Targets \filename{main.o} und füge in den Compiler-Aufrufen die Parameter \texttt{-MMD -MP} hinzu.
Binde außerdem die Dateien \filename{Building.d} und \filename{main.d} ein wie unten dargestellt:
\begin{lstmake}
all: main.exe

main.exe: main.o Building.o
    g++ -o main.exe main.o Building.o

main.o: main.cpp
    g++ -c -MMD -MP -o main.o main.cpp

Building.o: Building.cpp
    g++ -c -MMD -MP -o Building.o Building.cpp

-include Building.d main.d
\end{lstmake}

Um den Effekt dieser Lösung zu sehen, müssen wir alle generierten Dateien löschen (\filename{main.exe, Building.h, Building.cpp}).
Das anschließende Bauen sollte nun funktionieren.
Der Trick ist, dass \texttt{g++} beim Kompilieren für jede Sourcedatei ein Makefile generiert, das dessen eingebundene Header als Abhängigkeiten enthält (\filename{main.d, Building.d}).

Wenn du jetzt Änderungen an der \lstinline{toString}-Methode durchführst, werden diese anhand des Zeitstempels von \filename{Building.h} erkannt.

\subsection{Target \lstinline{clean}}

Bisher mussten wir hin und wieder die kompilierten Dateien manuell löschen, wenn wir unser Projekt neu bauen wollten.
Diese Aufgabe lässt sich mittels \texttt{make} ebenfalls automatisieren. Lege dazu ein neues Target \texttt{clean} ohne Abhängigkeiten an und füge einen entsprechenden Befehl zum Löschen ein:
\begin{lstmake}
clean:
    rm -rf main.o Building.o main.d Building.d main.exe

.PHONY: clean
\end{lstmake}
Das Spezial-Target \texttt{.PHONY} dient dazu, \texttt{make} zu signalisieren, dass \texttt{clean} keine Datei ist, die gebaut werden soll.
Würden wir dieses Target auslassen und eine Datei mit Namen \texttt{clean} erzeugen, würde \texttt{make} die Regel nie ausführen, weil die Datei ja existiert und keine Abhängigkeiten besitzt.
Du solltest \texttt{all} ebenfalls als \texttt{.PHONY} deklarieren.
Probiere es ruhig aus!

Um \texttt{clean} auszuführen klicke neben dem Build-Symbol auf den Auswahlpfeil und wähle \textbf{Project only - Clean} aus. In der Konsole siehst du die Ausgabe von \texttt{make clean}.

Es ist auch möglich, andere Targets als \texttt{all} oder \texttt{clean} auszuführen. Diese müssen in CodeLite allerdings erst eingerichtet werden: \textbf{Rechtsklick auf das Projekt} $\to$ \textbf{Settings...} $\to$ \textbf{Customize}. Klicke hier auf New... und gib einen Namen für den Target ein und den auszuführenden Befehl (z.B. \texttt{make main.o} um nur die Datei \texttt{main.cpp} zu kompilieren aber nicht zu linken).

\subsection{Generisches Compiler-Target}

Dir ist sicherlich aufgefallen, dass wir zwei Targets haben, die mehr oder weniger identisch sind: \filename{main.o} und \filename{Building.o}.

\texttt{make} bietet für solche Situationen generische Regeln an, die mittels Wildcards beschrieben werden.

Ersetze die beiden spezifischen Targets durch folgendes generisches:
\begin{lstmake}
%.o: %.cpp
    g++ -MMD -MP -c $< -o $@
\end{lstmake}
Die etwas kryptischen Ausdrücke \lstinline{\$<} und \lstinline{\$@} werden durch die aktuelle Abhängigkeit und Target ersetzt.

Lösche alle automatisch generierten Dateien (\texttt{make clean}) und baue das Projekt neu.

\subsection{Variablen in make}

Im Moment sieht unser Makefile in etwa so aus:
\begin{lstmake}
all: main.exe

main.exe: main.o Building.o
    g++ -o main.exe main.o Building.o

%.o: %.cpp
    g++ -MMD -MP -c $< -o $@

-include Building.d main.d

clean:
    rm -rf main.o Building.o main.d Building.d main.exe
    
.PHONY: clean all
\end{lstmake}

Dir ist sicherlich eine andere Form der Redundanz aufgefallen:
Noch immer haben wir die Tatsache, dass es im Moment zwei Sourcedateien gibt, an unterschiedlichen Stellen im Makefile festgelegt.
Wenn wir nun als nächstes die \lstinline{Floor}-Klasse entwerfen, müssten wir diese hinzufügen
\begin{itemize}
    \item als Abhängigkeit von \filename{main.exe},
    \item zum Linker-Aufruf in \filename{main.exe},
    \item in der \lstinline{-include}-Direktive und
    \item im Target \lstinline{clean}, und das gleich doppelt!
\end{itemize}
Das ist natürlich immer noch ziemlich fehleranfällig.

Wir würden also gerne nur an \emph{einer} Stelle definieren, welche Sourcedateien Teil unseres Projektes sind.

Da die Sourcedateien nirgendwo im Makefile auftreten, fangen wir mit den Objekt-Dateien an.
Lege am Anfang des Makefiles eine Variable mit dem Inhalt '\lstinline{main.o Building.o}' und ersetze das Auftreten der beiden Objekt-Dateien mit dieser Variablen:

\begin{minipage}{\textwidth}
\begin{lstmake}
OBJECTS=main.o Building.o

all: main.exe

main.exe: $(OBJECTS)
    g++ -o main.exe $(OBJECTS)
    
# and so on...
\end{lstmake}
\end{minipage}

Wiederhole die Prozedur für die Abhängigkeiten ('\lstinline{DEPEND=main.d Building.d}') und das ausführbare Programm ('\lstinline{BINARY=main.exe}').
Lasse dein Programm zwischendurch immer wieder vollständig neu bauen, um sicherzustellen, dass nichts kaputt geht.

Am Ende sollte dein Makefile in etwa so aussehen:
\begin{lstmake}
BINARY=main.exe
OBJECTS=main.o Building.o
DEPEND=main.d Building.d

all: $(BINARY)

$(BINARY): $(OBJECTS)
    g++ -o $(BINARY) $(OBJECTS)

%.o: %.cpp
    g++ -MMD -MP -c $< -o $@

-include $(DEPEND)

clean:
    rm -rf $(OBJECTS) $(DEPEND) $(BINARY)
    
.PHONY: clean all
\end{lstmake}
%$ for syntax highlighting

Wie wir die verbliebene Redundanz auflösen, sehen wir in der nächsten Teilaufgabe.

\subsection{Wildcard-Ausdrücke}

Die beiden Variablen \lstinline{OBJECTS} und \lstinline{DEPEND} sind strukturell ähnlich -- wieder etwas, das wir loswerden wollen.
Außerdem wäre es doch viel schöner, an einer Stelle die Sourcedateien zu definieren, oder?

Lege dazu eine neue Variable \lstinline{SOURCES} mit den beiden Sourcedateien an.
Der folgende Snippet zeigt, wie man nun mittels Suffix-Ausdrücken die anderen beiden Variablen erzeugt:
\begin{lstmake}
BINARY  = main.exe
SOURCES = main.cpp Building.cpp
OBJECTS = $(patsubst %.cpp, %.o, $(SOURCES))
DEPEND  = $(patsubst %.cpp, %.d, $(SOURCES))
\end{lstmake}

Es geht sogar noch allgemeiner:
Du kannst per regulärem Ausdruck\footnote{\url{https://de.wikipedia.org/wiki/Regul\%C3\%A4rer_Ausdruck}} definieren, dass \emph{alle} Sourcedateien im aktuellen Ordner verwendet werden sollen, indem du \lstinline{SOURCES} wie folgt definierst:

\begin{lstmake}
SOURCES=$(wildcard ./*.cpp)
\end{lstmake}
%$ for syntax highlighting

\subsection{Zusammenfassung}

Das Produkt unserer Bemühungen in dieser Aufgabe ist ein Makefile, das unabhängig davon ist, wie viele Sourcedateien du im aktuellen Verzeichnis hältst und wie sie genau heißen -- wichtig ist nur die Endung \lstinline{cpp}.

Hier nochmal das vollständige Makefile:
\begin{lstmake}
BINARY  = main.exe
SOURCES = main.cpp Building.cpp
OBJECTS = $(patsubst %.cpp, %.o, $(SOURCES))
DEPEND  = $(patsubst %.cpp, %.d, $(SOURCES))

all: $(BINARY)

$(BINARY): $(OBJECTS)
    g++ -o $(BINARY) $(OBJECTS)

%.o: %.cpp
    g++ -MMD -MP -c $< -o $@

-include $(DEPEND)

clean:
    rm -rf $(OBJECTS) $(DEPEND) $(BINARY)

.PHONY: clean
\end{lstmake}
%$ for syntax highlighting

\subsection{Nachwort}

Dies hier sind nur äußerst wenige der Möglichkeiten, die \texttt{make} bietet.%
\footnote{Für einen besseren Eindruck, sieh dir die Doku an: \url{https://www.gnu.org/software/make/manual/html_node/index.html}}
%
In der Praxis existieren Build-Tools, die eine wesentlich besser zu verstehende Beschreibungssprache verwenden und daraus Makefiles generieren.
Beispiele sind \texttt{cmake}\footnote{\url{http://www.cmake.org/cmake/help/cmake_tutorial.html}} oder \texttt{qmake}\footnote{\url{http://qt-project.org/doc/qt-4.8/qmake-tutorial.html}} (Bestandteil von Qt).
