\section{\ExercisePrefixAdvanced UnitTest++ \optional}\label{sec:unit_test}
\optionaltextbox
\cpppSolutionName{generic_linked_list_test}{generic\_linked\_list\_test}

In dieser Aufgabe wollen wir UnitTest++ benutzen um das Testen unseres Codes zu automatisieren.
Zuerst musst du dafür UnitTest++ herunterladen und wie in \url{http://codelite.org/LiteEditor/UnitTestPP } beschrieben bauen.
Nun sollst du deine Lösung zu Aufgabe \ref{sec:genericLinkedList} Generische Verkettete Liste testen. 
Notfalls kannst du auch die Lösung unter folgendem Pfad benutzen: \cpppLinkToSampleSolution{generic_linked_list}{generic\_linked\_list}. 
Erstelle nun ein neues UnitTest-Projekt. Füge zum Include Pfad unter Project/Settings den Ordner zu generic linked list ein. Falls du Funktionen aus cpp-Dateien testen möchtest, ist es in CodeLite am einfachsten, einen virtuellen Ordner im TestProjekt zu erstellen und mit Rechtsklick auf diesem die existierenden cpp-Dateien Dateien hinzuzufügen.

Erstelle nun 4 Tests indem du CHECK\_EQUAL und CHECK\_THROW (für Exceptions) verwendest. Dazu erzeugst du dir für jeden Test ein passendes List-Objekt.
 
\cpppInputListing{04_advanced/problems/listings/unittests.cpp} 
 

\hints{
	\item Unter \url{https://github.com/unittest-cpp/unittest-cpp/wiki/Macro-and-Parameter-Reference} kannst du weitere Makros für die Testerstellung finden.
}
