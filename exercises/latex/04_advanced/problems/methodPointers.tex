\subsection{\ExercisePrefixAdvanced Methodenzeiger \optional}
\optionaltextbox
\cpppSolutionName{functional_programming}{functional\_ programming}
\label{sec:functional_method}
Es gibt auch die Möglichkeit, Methoden\footnote{Hier ist eine einfache Erklärung zu dem Unterschied von Funktion und Methode zu finden \url{http://stackoverflow.com/a/155655}} via Zeiger auszuführen (sogenannte \textbf{Methodenzeiger}).
Hierzu muss man zusätzlich zu dem Funktionsnamen noch ein Objekt übergeben, auf das die Methode angewendet wird.

In unserem Beispiel aus Aufgabe \ref{sec:functional} von \lstinline{Square} fügen wir noch eine weitere Methode \lstinline{squareroot} hinzu, welche das Inverse der Quadrierung ausführt.
Unsere Klasse Square verändert sich dementsprechend zu
%
\cpppInputListing{04_advanced/problems/listings/functional_method_square.cpp}
%
und unser \lstinline{map} zu 
%
\cpppInputListing{04_advanced/problems/listings/functional_method_map.cpp}
%
Deine Aufgabe ist es nun, eine neue Implementation von \lstinline{map} hinzuzufügen, deinem für \lstinline{map} geschriebenen Funktor eine weitere Methode hinzuzufügen und anschließend deine Implementation mit allen implementierten Methoden zu testen.