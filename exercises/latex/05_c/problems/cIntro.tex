\section{\ExercisePrefixEmbeddedC Die Programmiersprache C im Vergleich zu C++}
\label{sec:CVersusCPlusPlus}

In den nächsten Tagen werden wir Programme für eine Embedded-Plattform in C entwickeln.
Da C++ aus C entstand, sind viele Features von C++ nicht in C enthalten. Im Folgenden sollen die Hauptunterschiede verdeutlicht werden.
%
\begin{itemize}
	\item Keine OO-Konzepte (Vererbung, \dots)
    \item Strukturen (\lstinline{struct}) statt Klassen (\lstinline{class})
	\item Keine Templates
	\item Keine Referenzen, nur Zeiger und Werte
	\item Kein \lstinline{new} und \lstinline{delete}, sondern \lstinline{malloc()} und \lstinline{free()} (\lstinline|#include <stdlib.h>|)
	\item Je nach Sprachstandard müssen Variablen am Anfang der Funktion deklariert werden (Standard-Versionen bis einschießlich C99)
	\item Parameterlose Funktionen müssen \lstinline{void} als Parametertyp haben, leere Klammern (bspw. \lstinline{int foo();}) bedeuten, dass beliebige Argumente erlaubt sind.
	\item Keine Streams, stattdessen \lstinline{(f)printf} zur Ausgabe auf Konsole und in Dateien (\verb|#include <stdio.h>|)
	\item Kein \lstinline{bool}-Datentyp, stattdessen wird 0 als \lstinline{false} und alle anderen Zahlen als \lstinline{true} gewertet
	\item Keine Default-Argumente
	\item Keine \lstinline{std::string} Klasse, nur \lstinline{char}-Arrays, die mit dem Nullbyte (\lstinline{'\0'}) abgeschlossen werden.
	\item Keine Namespaces
\end{itemize}
%
Da einige dieser Punkte sehr entscheidend sind, werden wir auf diese im Detail eingehen.
Wichtig ist hierbei, dass alle im Folgenden vorgestellten Konzepte auch in C++ zur Verfügung stehen.
Die Header der C-Bibliothek sind alle auch in C++ verfügbar.
Möchte man bspw. \lstinline{malloc} nutzen, kann man dies in C++ über \lstinline|#include <stdlib.h>| oder über \lstinline{#include <cstdlib>} (Allgemeines Muster: vorangestelltes 'c' und fehlendes '.h').
Im zweiten Fall sind alle Funktionen im Namensraum \lstinline{std} eingebettet, man muss als \lstinline{std::malloc} nutzen.

\subsection{Kein OO-Konzept}
In C gibt es keine Klassen, weshalb die Programmierung in C eher Pascal statt C++ ähnelt.
Stattdessen gibt es Strukturen (\lstinline{struct}), die mehrere Variablen zu einem Datentyp zusammenfassen, was vergleichbar mit Records in Pascal oder -- allgemein -- mit Klassen ohne Methoden und ohne Vererbung ist.
%
Die Syntax dafür lautet
%
  \lstinputlisting{05_c/problems/listings/cintro_structs.c}
%
Zum Beispiel
%
  \lstinputlisting{05_c/problems/listings/cintro_structs_ex.c}
%
Die Sichtbarkeit aller Attribute ist automatisch \lstinline{public}.
%
Um den definierten \textbf{\lstinline|struct|} als Datentyp zu verwenden, muss man zusätzlich zum Namen das Schlüsselwort \lstinline{struct} angeben:
%
  \lstinputlisting{05_c/problems/listings/cintro_structs_func.c}
%
Um den zusätzlichen Schreibaufwand zu vermeiden, wird in der Praxis oft ein \lstinline{typedef} auf den \lstinline{struct} definiert:
%
  \lstinputlisting{05_c/problems/listings/cintro_structs_typedef.c}
%
Man kann die Deklaration eines \lstinline{struct} auch direkt in den \lstinline{typedef} einbauen:
%
  \lstinputlisting{05_c/problems/listings/cintro_structs_short.c}

\subsection{Kein \lstinline{new} und \lstinline{delete}}

Anstelle von \lstinline{new} und \lstinline{delete} werden die Funktionen \lstinline{malloc} und \lstinline{free} verwendet, um Speicher auf dem Heap zu reservieren.
Diese sind im Header \filename{stdlib.h} deklariert.

  \lstinputlisting{05_c/problems/listings/cintro_mallocfree.c}

\subsection{Ausgabe auf Konsole per \lstinline{printf}}

Um Daten auf der Konsole auszugeben, kann die Funktion \lstinline{printf} verwendet werden.
\lstinline{printf} nimmt einen Format-String sowie eine beliebige Anzahl weiterer Argumente entgegen.
Der Format-String legt fest, wie die nachfolgenden Argumente ausgegeben werden.
Mittels \textbackslash n kann man einen Zeilenvorschub erzeugen. Um \lstinline{printf} zu nutzen, muss der Header \filename{stdio.h} eingebunden werden.\footnote{Weitere mögliche Parameter etc. der Funktion \lstinline{printf} findest du unter \url{http://www.cplusplus.com/reference/cstdio/printf/}.}
Der folgende Codeausschnitt zeigt, wie man Zahlen und Zeichen mit \lstinline|printf| ausgeben kann.
Wenn die Variablen \lstinline|i| und \lstinline|c| nicht definiert werden, ist die Aufgabe undefiniert.
Zu beachten ist auch, dass \lstinline|printf| nicht automatisch einen Zeilenumbruch einfügt.
%
  \lstinputlisting{05_c/problems/listings/cintro_printf.c}

Im Folgenden werden wir untersuchen, welche Folgen es hat, wenn man das \enquote{per Konvention} erwartete Null-Byte (\lstinline|'\0'|) entfernt.
In diesem Fall wird der Speicher Byte-weise solange ausgegeben, bis ein Null-Byte angetroffen wird.
\begin{itemize}
\item 
Beginne mit einer leeren \lstinline|main|-Funktion.
\item 
Lege einen Puffer der Größe 6 an:
\begin{lstlisting}
char *buffer = malloc(6 * sizeof(char));
\end{lstlisting}
\item 
Kopiere mittels \lstinline|strcpy| (aus dem Header \lstinline|string.h|) den String \lstinline|"Hello"| in den Puffer:
\begin{lstlisting}
strcpy(buffer, "Hello");
\end{lstlisting}
\item 
Nun gib den Inhalt des Puffers mittels \lstinline|printf| aus:
\begin{lstlisting}
printf("%s\n", buffer);
\end{lstlisting}
Wenn du das Programm jetzt kopilierst und ausführst, sollte nur der String \lstinline|Hello| erscheinen.
\item 
Jetzt beginnt der spannende Teil:
Überschreibe das Null-Byte mit einem beliebigen Zeichen (bspw. \lstinline|'_'|).
\begin{lstlisting}
buffer[5] = '_';
\end{lstlisting}
Wenn du jetzt den Inhalt des Puffers ausgibst, kann alles passieren (\enquote{Undefined Behavior}).
Die Funktion \lstinline|printf| wird solange den Speicher auslesen, bis sie ein Null-Byte trifft.
Dieses Experiment zeigt, dass es sehr wichtig ist, bei der Manipulation von C-Strings gut aufzupassen -- 
zahlreiche Sicherheitslücken basieren auf dieser Schwäche!
\item
Ändere zum Abschluss die Art, wie der Puffer allokiert wird wie folgt:
\begin{lstlisting}
char *myString = "Hello";
\end{lstlisting}
Jetzt liegt der Puffer nicht mehr auf dem Heap (wie bei \lstinline|malloc|) sondern im \texttt{data}-Segment.
Wenn du den Code nun kompilierst und ausführst, solltest du einen Speicherzugriffsfehler (\enquote{Segmentation Fault}) erhalten, da Schreibzugriffe in diesem Speicherbereich verboten sind.
\end{itemize}

Es ist auch möglich, einen C-String \emph{einzukürzen}, indem man das Null-Byte verschiebt innerhalb des Puffers nach vorne verschiebt.
\begin{itemize}
\item
Verwende den Code aus dem vorherigen Abschnitt und diejenige Zeile an, in der der Puffer manipuliert wird.
Statt \lstinline|'_'| an Index 5 zu platzieren, platziere jetzt das Null-Byte (\lstinline|'\0'|) an Index 2.
\item 
Bei der Ausführung sollte jetzt nur noch \texttt{He} ausgegeben werden.
\end{itemize}

\subsection{Strings zusammenbauen}

Die Funktion \lstinline{sprintf} dient dazu, formatierte Strings zusammenzusetzen und ist syntaktisch eng mit der Funktion \lstinline{printf} verwandt.\footnote{Weitere mögliche Parameter etc. der Funktion \lstinline{sprintf} findest du unter \url{http://www.cplusplus.com/reference/cstdio/sprintf/}.}
%
\begin{itemize}
\item
Beginne mit einer leeren \lstinline|main|-Funktion.
\item 
Lege erneut einen Puffer auf dem Heap an:
\begin{lstlisting}
char *buffer = malloc(100 * sizeof(char));
\end{lstlisting}
\item 
Gib den Puffer direkt mittels \lstinline|printf| aus:
\begin{lstlisting}
printf(buffer);
\end{lstlisting}
Dies zeigt, dass es sehr gefährlich ist, mit uninitialisierten C-Strings zu hantieren.
Sicherer wäre es, direkt nach der Allokation ein Null-Byte an Index 0 des Puffers zu platzieren:
\begin{lstlisting}
buffer[0] = '\0';
\end{lstlisting}
Wenn du anschließend erneut \lstinline|printf| aufrufst, sollte die Ausgabe leer bleiben.
\item 
Verwende den folgenden Aufruf, um einen Gruß mittels \lstinline|sprintf| zusammenzusetzen.
Lege dazu im Vorfeld eine Variable \lstinline|name| an.
\begin{lstlisting}
sprintf(buffer, "Hello, %s!\n", name);
\end{lstlisting}
\item 
Den so gefüllten Puffer gibst du wie gewohnt über \lstinline|printf| aus:
\begin{lstlisting}
printf(buffer);
\end{lstlisting}
\item 
Die Funktion \lstinline|sprintf| kann natürlich auch zur Ausgabe (bspw.) von Zahlen und Zeichen genutzt werden:
\begin{lstlisting}
sprintf(buffer, "c = %c, i = %3d", c, i);
\end{lstlisting}
\end{itemize}
Beachte auch hier, dass der Puffer auf dem Heap (statt im \texttt{data}-Segment) liegen muss, damit die Funktion \lstinline|sprintf| hineinschreiben kann.
Andernfalls erhältst du einen Segmentation Fault.

\subsection{Zahlen formatiert ausgeben}
\label{sec:CFormatNumbers}
Schreibe ein C-Programm, welches alle geraden Zahlen von 0 bis 200 formatiert ausgibt.
Die Formatierung soll entsprechend dem Beispiel erfolgen:
%
  \lstinputlisting{05_c/problems/listings/cintro_format.c}
%
Mache die Spaltenzahl und Spaltenbreite mithilfe von Variablen konfigurierbar, sodass es auch leicht möglich ist, 15 Spalten und/oder Zahlen bis \numprint{10 000} auszugeben.

\subsection{Bit-Operatoren in C (und C++)}

In dieser Aufgabe machst du dich mit den Bit-Operatoren (\lstinline{&, |, ~, >>, <<}) in C vertraut.
Alle Operatoren können exakt gleich in C++ verwendet werden.
Bit-Operatoren sind für ganzzahlige Typen definiert (bspw. \lstinline|(unsigned) int|, \lstinline|(unsigned) char|).
Ein Bit-Operator bezieht sich dabei auf jedes einzelne Bit.
Im Gegensatz dazu beziehen sich logische Operatoren (\lstinline{||, &&}) immer auf den gesamten Wert.

Zum Experimentieren stellen wir dir eine Vorlage zur Verfügung: \url{https://github.com/Echtzeitsysteme/tud-cppp/blob/master/exercises/templates/d05t01_BitAndLogicOperations/d05t01_BitAndLogicOperations.c}.
Die Funktion \lstinline|fmt| wandelt ein Byte in einen String um, der das Bit-Pattern darstellt.
Zum Beispiel ist die Ausgabe von \lstinline|fmt(23)| der String \lstinline|"0b00010111"| ($19 = 1 + 2 + 4 + 16$).
\begin{itemize}
\item 
Fülle zunächst die mit \lstinline|// TODO implement me| gekennzeichneten Zeilen aus, sodass die Ausgabe korrekt ist.
Beispiele sind für \texttt{NEG} und \texttt{NOT} gegeben.

Stelle sicher, dass deine Ergebnisse für \lstinline|a = 23, b = 3| mit den erwarteten Ergebnisse in folgender Tabelle übereinstimmen\\
\begin{tabular}{lr}
\toprule
\textbf{Ausdruck} & \textbf{Ergebnis}\\
\midrule
\texttt{a AND b} & 3\\
\texttt{a OR b} & 23\\
\texttt{a XOR b} & 20\\
\midrule
\texttt{a RIGHT s} & 5\\
\texttt{a LEFT s} & 92\\
\bottomrule
\end{tabular}

Die folgende Tabelle enthält die erwarteten Ergebnisse der logischen Operatoren für 
\begin{inparaenum}
\item \lstinline|a = 1, b = 1|
\item \lstinline|a = 1, b = 0|
\end{inparaenum} :\\
\begin{tabular}{lrrrr}
    \toprule
    \textbf{Ausdruck} & \multicolumn{2}{c}{\textbf{Ergebnis}}\\
    & \lstinline|a=1,b=1| & \lstinline|a=1, b=0| & \lstinline|a=0, b=1| & \lstinline|a=0, b=0|\\
    \midrule
    \texttt{a LAND s}  & 1  & 0& 0& 0\\
    \texttt{a LOR s}   & 1  & 1& 1& 0\\
    \texttt{a XOR s}   & 0  & 1& 1& 0\\
    \texttt{a IMP s}   & 1  & 0& 1& 1\\
    \texttt{a BIIMP s} & 1  & 0& 0& 1\\
    \bottomrule
\end{tabular}
\item 
Im Allgemeinen entspricht eine Verschiebung nach links/rechts einer Multiplikation mit/Division durch 2.
Dazu werden jeweils von rechts/links 0-Bits eingeschoben.
Unter welchen Umständen gilt dies nicht mehr?
Was passiert, wenn du den Datentyp von \lstinline|a| auf \lstinline|unsigned char| setzt?
Beobachte auch, ob das angezeigte Bitmuster mit dem ausgegebenen Wert übereinstimmt.
% << Überlauf 127/255 zu 0, aber der Ausdruck "a << s" wird automatisch in den nächsthöheren Typ konvertiert!
% >> Anrunden bei nicht durch 2^s teilbaren Werten
\item 
Ändere den Wert der Variablen \lstinline|a| in \lstinline|-1|.
Was passiert bei einem Rechts-Shift?
Was bei einem Links-Shift?
Wie unterscheidet sich das Verhalten im Vergleich zu positivem \lstinline|a|?
% Es werden '1' statt '0' eingeschoben
\item 
Ändere den Wert von \lstinline|s| zu -1.
Wie sieht das Ergebnis des Rechts-/Links-Shifts aus?
Ein Shift um einen negativen Wert ist \enquote{Undefined Behavior}.
Welche Ergebnisse sind demnach erlaubt?
\end{itemize}

\subsection{C++-Aufgaben nachprogrammieren \optional}

\optionaltextbox

Dies ist eine freie Aufgabe, in der du versuchst, Programme der vergangenen Tage in reinem C auszudrücken.
Der Schwierigkeitsgrad ist dabei sehr unterschiedlich!

\paragraph{Niedrigerer Schwierigkeitsgrad}
\begin{itemize}
\item
\textbf{Sternen- oder Buchstabenmuster} ausgeben (Blatt 1):
Diese Aufgabe ist sehr ähnlich zu \cref{sec:CFormatNumbers}.

\item
\textbf{Werte analysieren} und mit \textbf{Arrays} arbeiten (Blatt 2):
Abgesehen von der fehlenden Unterstützung für Referenzen sollten die Ergebnisse sich nicht unterscheiden.

\item
\textbf{Funktionszeiger} (Blatt 4):
Funktionszeiger in C arbeiten genauso wie Funktionszeiger in C++.
Du kannst dies testen, indem du die Programmlogik der vorigen Aufgabe in eine separate Funktion auslagerst, die neben der Spaltenbreite und Obergrenze der anzuzeigenden Zahlen zusätzlich einen Funktionszeiger-Parameter hat, der festlegt, was mit der jeweiligen Zahl vor der Ausgabe geschehen soll (bspw. verdoppeln, quadrieren).
\end{itemize}
\paragraph{Höherer Schwierigkeitsgrad}
\begin{itemize}
\item
\textbf{\lstinline|Vector|} (Blatt 1): C bietet keine Objektorientierung, aber du kannst eine \lstinline|struct| zur Datenhaltung anlegen und die Methoden der Klasse als Funktionen realisieren, deren Namen bspw. immer mit dem Präfix \lstinline|Vector_| beginnen und die als zusätzlichen Parameter einen Pointer auf eine \lstinline|Vector|-\lstinline|struct| erhalten.

\item 
\textbf{Verkettete Listen} (Blatt 2):
Die reinen Datenstrukturen für die Liste (\lstinline|List|) und für Listenelemente (\lstinline|ListItem|) lassen sich als \lstinline|struct| abbilden.
Methoden kannst du wieder auf Funktionen mit Namenskonvention und zusätzlichem Pointer-Parameter abbilden.

\item
\textbf{Generischer Vektor und Liste} (Blatt 4):
Auch wenn es in C keinen eingebauten Mechanismus wie die C++-Templates gibt, kannst du die \lstinline|Vector| und \lstinline|List|-Klasse generisch machen, indem du die Einträge des Vektors bzw. den \lstinline|content| der \lstinline|ListItems| als \lstinline|void*| deklarierst.

\item
\textbf{Eigene \lstinline|Array|-Klasse} (Blatt 4):
Ebenso wie bei generischem Vektor und Liste kannst du natürlich auch deine eigene \lstinline|Array|-Klasse schreiben.
\end{itemize}
