\section{\ExercisePrefixEmbeddedC Taster abfragen \optional}

In dieser Aufgabe erweiterst du die vorherige Aufgabe um eine Benutzerinteraktion über den Taster des linken Joysticks (\textbf{Joystick 1}).
Der Taster ist als digitaler Eingang am \textbf{Pin F5} des Mikrocontrollers angeschlossen.
Ziel dieser Aufgabe ist es, mithilfe des Tasters die RGB-LED in zwei verschiedenen Szenarien zu kontrollieren.
Im ersten Szenario soll der Taster als Lichtschalter arbeiten: Wird der Taster einmal betätigt, schaltet sich die blaue LED ein; wird der Taster erneut betätigt, schaltet sie sich wieder ab.
Im zweiten Szenario soll die blaue LED solange leuchten, wie der Taster gedrückt gehalten wird.

\begin{enumerate}
\item 
In dieser Aufgabe wirst du mit den Dateien \filename{button.c} und \filename{button.h} arbeiten.

\item 
Um den Zustand der aktuellen LED zu speichern
Zunächst werden wir die nötigen Variablen in \textbf{button.c} deklarieren:
Um die blaue LED ansteuern zu können, benötigst du einen Zeiger auf das Daten-Register der blauen LED (\textbf{\lstinline|blueLEDValuePort|}).
Weiterhin soll für den Lichtschalter der aktuelle Zustand der LED in einer vorzeichenlosen 8-Bit-Integer-Variablen \textbf{\lstinline|ledStatus|} gespeichert werden.

\item
Implementiere zunächst die Funktion \textbf{\lstinline|initLed()|}.
Diese soll \lstinline|ledStatus|, den Pin F5 und die blaue LED initialisieren.
\begin{itemize}
\item 
Der LED-Status soll zu Beginn 0 (= \enquote{aus}) sein.
\item 
Das Daten-Register der blauen LED soll entsprechend initialisiert sein, dass die LED angeschaltet ist.
\item 
und Pin F5 soll durch die PDL-Methode \textbf{\lstinline|Gpio1pin_InitIn(pin, option)|} initialiseren werden.
Diese Funktion ist eine weitere Möglichkeit Pins des Mikrocontroller zu initialiseren. 
\textbf{\lstinline|Gpio1pin_InitIn|} intialisiert einen Pin als Eingang und kann zusätzlich zu diesem einen Pull-Up-Widerstand schalten, um das ankommende Signal zu verstärken.
Der Name des Pins wird mit \textbf{\lstinline|GPIO1PIN_PF5|} angegeben.
Der Pull-Up-Widerstand wird mit der Option \textbf{\lstinline|Gpio1pin_InitPullup(1u)|} aktiviert.
\end{itemize}

\item
Implementiere nun die Funktion \textbf{\lstinline|toggleBlueLED()|}.
Diese den Status von \lstinline|ledStatus| umkehren:
War der Wert zuvor \lstinline|1|, soll er danach \lstinline|0| sein und umgekehrt.
Der aktuelle Status der LED kann mit \textbf{\lstinline|setBlueLED(uint8_t status)|} gesetzt werden. 

\item 
Implementiere nun die Funktion \textbf{\lstinline|isButtonPressed|}, die zurückgibt, ob der Taster gerade gedrückt ist.
Dazu kannst du die PDL-Funktion \lstinline|Gpio1pin_Get(pin)| nutzen.
Diese Funktion liefert den Wert des übergebenen Pins zurück (hier: \lstinline|GPIO1PIN_PF5|).
Beachte, dass der Taster durch den Pull-Up-Widerstand genau dann gedrückt ist, wenn am Pin ein niedriger Pegel (\lstinline|0|) anliegt.

\item
Implementiere nun mithilfe der zuvor erstellten Hilfsfunktionen die Hauptfunktionen \textbf{ButtonToggleBlueLED()} und \textbf{ButtonHoldBlueLEDOn()}.
\textbf{ButtonToggleBlueLED()} soll dem Button die Funktion eines Lichtschalters geben und  \textbf{ButtonHoldBlueLEDOn()} soll die LED zum Leuchten bringen, solange der Taster gedrückt gehalten wird. 

\end{enumerate}
\hints{	
	\item \textbf{\lstinline|while(Gpio1pin_Get(GPIO1PIN_PF5) == 0)|} kann verwendet werden, um das Programm solange zu pausieren, bis der Taster losgelassen wird. 
}
