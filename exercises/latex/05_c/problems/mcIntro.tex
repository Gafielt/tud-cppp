\section{Testprogramm auf den Microcontroller laden \optional}

\subsection{Überblick}
Für die Arbeit mit dem Evaluationsboard nutzen wir die Entwicklungsumgebung \emph{WinIDEA Open}\footnote{\url{http://www.isystem.com/download/winideaopen}}.

Im Vergleich zu CodeLite ist diese Umgebung speziell auf die Entwicklung von Embedded C zugeschnitten.
Der Bauprozess für Embedded-C-Programme sieht teilweise anders aus als bei C++-Programmen:
\begin{itemize}
\item 
Das Ergebnis der \textbf{Link-Phase} ist kein auf dem PC ausführbares Programm, sondern ein sogenanntes \emph{Image}
Dieses Image wird in den statischen Speicher des Microcontrollers geladen.
\item
Nach der Link-Phase folgt die \textbf{Flash-Phase}.
Während dieser Phase wird das Image auf den Microcontroller übertragen.
\item 
Anschließend beginnt die \textbf{Ausführung} direkt oder man muss den \emph{Reset}-Knopf des Boards drücken, um den Programmzähler zurückzusetzen.
\item
WinIDEA geht hierbei direkt in den \textbf{Debug-Modus}.
Das bedeutet, dass die Ausführung des Programms bei der ersten Instruktion angehalten wird.
\RKi{Screenshot?}
\end{itemize}

\subsection{Testprogramm}
Für diese und alle weiteren Aufgaben stellen wir dir ein Codetemplate zur Verfügung, das von dir ergänzt wird.
Wir beginnen mit einem kleinen fertigen Programm, das die RGB-LED des Evaluationsboards periodisch blinken lässt.
Dies ist das \enquote{Hello World}-Programm der Embedded-C-Welt.

\RKi{Schritte bis zum Laufenden Programm mit Icons/Beschreibung unterstützen}
\RKi{Quellcode des Programms einfügen und erklären}

Von der Toolseite aus unterscheidet sich die Entwicklung für den Mikrocontroller kaum von den bisherigen Übungen.
Der Aufruf der Toolchain (Compiler, Linker und Flashprogramm) für die Embedded-Entwicklung ist in einem Makefile integriert, das in CodeLite durch den Buildprozess automatisch aufgerufen wird.