\section{Template Funktionen}
\subsection{Templatefunktionen implementieren}
Implementiere die folgende Funktion, die das Maximum von zwei Variablen liefert:

\lstinputlisting{problems/listings/templates_max.hpp}

Durch die Verwendung von Templates soll die Funktion mit verschiedenen Datentypen funktionieren.
Teste deine Implementation.

In der Vorlesung haben wir gesehen, dass jede Verwendung von \lstinline{t1} und \lstinline{t2} in \lstinline{maximum} eine Schnittstelle induziert, die der Typ \lstinline{T} bereitstellen muss.
Das bedeutet, dass \lstinline{T} alle Konstruktoren, Methoden und Operatoren zur Verfügung stellen muss, die in \lstinline{maximum} genutzt werden.

Wie sieht diese Schnittstelle in diesem Fall aus?
Welche Gründe gibt es, den Rückgabewert der Funktion \lstinline{maximum} als konstante Referenz festzulegen?

\hints{
    \item In den meisten Fällen kann anstelle von \lstinline{typename} auch \lstinline{class} in der Template-Deklaration verwendet werden.
    
    \item In der Regel muss die Definition von Template-Funktionen und -Methoden im Header erfolgen.
    Allgemeiner: zur Compilezeit in der gleichen \filename{cpp}-Datei wie ihre Verwendung. 
    Das hängt damit zusammen, dass Templates sprichwörtlich nur Vorlagen sind, deren Typparameter zur Compilezeit mit den konkret verwendeten Typen ersetzt werden.
    Würde man Templates in separaten \filename{cpp}-Dateien implementieren, dann könnte die Verbindung zwischen der Verwendungsstelle und der Definition erst zur Linkzeit hergestellt werden -- also zu spät.
}

\subsection{Explizite Angabe der Typparameter}
Lege nun zwei Variablen vom Typ \lstinline{int} und \lstinline{short} an, und versuche, mittels \lstinline{maximum()} das Maximum zu bestimmen.
Der Compiler wird mit der Fehlermeldung \textbf{no matching function for call...} abbrechen, da er nicht weiß, ob \lstinline{int} oder \lstinline{short} der Template-Parameter sein soll.
Gib deshalb den Template-Parameter mittels \lstinline{maximum<int>()} beim Aufruf von \lstinline{maximum()} explizit an.
Die übergebenen Parameter werden dabei vom Compiler automatisch in den gewünschten Typ umgewandelt.

\subsection{Induzierte Schnittstelle implementieren}
Erstelle eine Klasse \lstinline{C}, die eine Zahl als Attribut beinhaltet. Implementiere einen passenden Konstruktor sowie einen Getter für diese Zahl. Nun wollen wir unsere Funktion  \lstinline{maximum()} verwenden, um zu entscheiden, welches von zwei \lstinline{C}-Objekten die größere Zahl beinhaltet.
Überlege dir, was zu tun ist, und implementiere es.

\hints{
	\item Die Klasse \lstinline{C} muss mindestens die durch \lstinline{maximum} induzierte Schnittstelle implementieren.
}
